\documentclass{cgyrf15}
% The packages graphicx, amssymb, and amsmath are already loaded via
% cgyrf15.cls

%----------------------- Macros and Definitions --------------------------

% Add all additional macros here, do NOT include any additional files.

% Nice table
\usepackage[table]{xcolor}
\usepackage{tabularx}
\usepackage{booktabs}
\definecolor{lightbluishgrey}{rgb}{0.9,0.91,0.95}
\newcommand{\ra}[1]{\renewcommand{\arraystretch}{#1}}
\newcommand{\NO}{{\color{red}\textsf{X}}}
\newcommand{\YES}{$\bullet$}
\newcommand{\alec}[1]{{\textcolor[rgb]{0.1,0.6,0.1}{\textbf{Alec:} #1}}}
\newcommand{\leo}[1]{{\textcolor[rgb]{0.8,0.0,0.0}{\textbf{Leo:} #1}}}


%----------------------- Title -------------------------------------------

\title{Nested Cages - Appendix}

\author{
Leonardo Sacht\thanks{Universidade Federal de Santa Catarina, Brazil.} \and
Alec Jacobson\thanks{Columbia University, USA.} \and
Etienne Vouga\thanks{University of Texas at Austin, USA.} 
}

\begin{document}

\maketitle

\begin{abstract}
This document provides additional results, timings and discussion
for our CG:YRF paper \emph{Nested Cages}.
\end{abstract}

\section{Additional results}

We implemented a prototype of our method as a serial \textsc{Matlab} program.
%
We report timings of our unoptimized code for a few representative examples in
Table~\ref{tab:timings} recorded on an iMac Intel Core i7 3.5GHz computer with 8GB
memory.
%
As expected the bottleneck is the collision-free re-inflation step.
%
We experimented with a wide variety of shapes, ranging from CAD models,
characters, and scanned objects (see Figure~\ref{fig:bigzoo}). By default we compute
layers so that each coarser layer has $2^{-\frac{2}{3}}$ times as many facets as
the previous finer layer, a ratio chosen so that resulting tetrahedral
meshes will have approximately $8$ times fewer elements. For most
meshes we compute seven layers, with fewer for lower resolution inputs.

\section{Failure Cases}

Our insight to break the multi-layer nesting problem into pairwise subproblems
ensures tractability, but in some cases leads to converging at an
``artificial local minimum.'' If a coarse cage collides with itself during
inflation then it may create a pinch that blocks inflation of subsequent coarse
layers (see Figure~\ref{fig:homer}).
%
One solution is to iterate through the fine layers to make sufficient room in
these problem areas, but defining this relaxation direction is not obvious.

In cases where the input coarse cage begins too far away from the fine mesh,
the signed-distance flow will fail: for example, by flowing vertices of a
triangle into opposite parts of the coarse mesh. 
%
Adding a small amount of smoothing on the flowing fine mesh or expanding the
coarse mesh alleviates some of these problems, but a general solution is
allusive.
%
The correct assignment seems related to correctly matching medial axes of both
meshes. Perhaps this is an avenue of future improvement.

\begin{table}[t]
\centering
\ra{1.2}
\setlength{\tabcolsep}{5.5pt}
\rowcolors{2}{white}{lightbluishgrey}
\begin{tabular}{l r r r r l r r r r r r r r r r r r}
\rowcolor{white}
Name  & \#F & avg($t_{flow}$) & avg($t_{re}$) & Energy\\
\midrule
Gargo & 13,500  & 2s & 90s & ARAP\\
Warrior & 26,658  & 2s & 86s & ARAP  \\
Pelvis & 40,316  & 11s & 460s & Volume  \\
Bunny & 52,910  & 11s & 202s & ARAP \\
Mug & 74,720  & 7s & 54s & Volume \\
Octopus & 100,000  & 11s & 227s & Volume \\
\bottomrule
\end{tabular}
\caption{
We report the average time per cage to flow ``avg($t_{flow}$),'' and to
re-inflate ``avg($t_{re}$)''.}
\label{tab:timings}
\end{table}

\begin{figure}
  \includegraphics[width=\linewidth]{figs/homer-fail}
  \caption{If a expanding coarse mesh collides with itself (green), it creates
  a \emph{pinch} preventing processing of further coarser layers.}
  \label{fig:homer}
\end{figure}

\begin{figure*}[t]
  \includegraphics[width=\linewidth]{figs/zoo_big}
  \caption{Each triplet shows: input model, slice through 
  all nested layers, and outermost, coarsest layer. 
  }
  \label{fig:bigzoo}
\end{figure*}

%Use of Bibtex is not necessary, but strongly recommended.
%Please make sure that references are displayed in a uniform fashion
\bibliographystyle{abbrv}
\bibliography{references}
 
\end{document}


