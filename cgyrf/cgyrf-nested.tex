\documentclass{cgyrf15}
% The packages graphicx, amssymb, and amsmath are already loaded via
% cgyrf15.cls

%----------------------- Macros and Definitions --------------------------

% Add all additional macros here, do NOT include any additional files.

% Nice table
\usepackage[table]{xcolor}
\usepackage{tabularx}
\usepackage{booktabs}
\definecolor{lightbluishgrey}{rgb}{0.9,0.91,0.95}
\newcommand{\ra}[1]{\renewcommand{\arraystretch}{#1}}
\newcommand{\NO}{{\color{red}\textsf{X}}}
\newcommand{\YES}{$\bullet$}


%----------------------- Title -------------------------------------------

\title{Nested Cages}

\author{
Leonardo Sacht\thanks{Universidade Federal de Santa Catarina, Brazil.} \and
Alec Jacobson\thanks{Columbia University, USA.} \and
Etienne Vouga\thanks{University of Texas at Austin, USA.} 
}

\begin{document}

\maketitle

\begin{abstract}
Many tasks in geometry processing and physical simulation benefit from multiresolution hierarchies: efficient PDE solving, collision detection culling, physically based simulation, and cage-based geometric modeling. We propose an algorithm to generate coarse domain approximations that satisfy several desiderata for efficient and accurate multiresolution algorithms: the coarsened domains respect the original topology, enclose the original object, yet fit as tightly as possible.
\end{abstract}

\section{Introduction}

As the complexity and size of computational objects continue to grow,
acceleration algorithms become increasingly important. One powerful technique
is to decompose a high-resolution mesh into a hierarchy of increasingly coarse
approximations or \emph{cages} (see Figure~\ref{fig:horse}). For example,
multiresolution FEM techniques efficiently solve Eulerian PDEs on very fine
meshes by moving up and down the hierarchy.
Coarse enclosing cages also find use in physical simulation,
where deformations of the cage are interpolated onto embedded high-resolution
geometries; in interactive animation, where artists specify large-scale
deformations by adjusting the low-dimensional cage; and in collision detection,
where conservative culling reduces computation time. For all these
applications, the key to high performance is the ability to generate a quality
multiresolution.

Existing techniques such as surface mesh decimation, voxelization, 
or contouring distance level sets, violate one or more of three characterstics: 
strict nesting, homeomorphic topology and geometric closeness
(Table~\ref{tab:feature-chart}). 
Instead, our methods successively constructs each next-coarsest 
level of the hierarchy, using a sequence of decimation, flow, 
and contact-aware optimization steps. 
From coarse to fine, each layer then fully encages the next while retaining a 
snug fit and respecting the original surface topology.

\begin{figure}[t]
  \includegraphics[width=\linewidth]{figs/horse-25-layers}
  \caption{We generate 25 homeomorphic tight nesting cages 
  for the \emph{Horse} mesh.}
  \label{fig:horse}
\end{figure}

\begin{table}
\centering
\ra{1.2}
\setlength{\tabcolsep}{5.5pt}
\rowcolors{2}{white}{lightbluishgrey}
\begin{tabularx}{\linewidth}{X c c c}
\rowcolor{white}
\toprule
Method               & \textsc{Homeom.} & \textsc{Nesting}& \textsc{Tight} \\
\midrule                                                                       
\textbf{Our method}  & \YES                  & \YES            & \YES           \\
\emph{Decimation}                                                               
                     & \YES                  & \NO             & \YES           \\
\emph{Voxelization}                                                             
                     & \NO                   & \YES            & \NO            \\
\emph{Level-set}                                                                
                     & \NO                   & \YES            & \NO            \\
\bottomrule
\end{tabularx}
\caption{Hierarchy construction feature chart: Previous techniques fall short
on one property or another, while ours satisfies all of them.}
\label{tab:feature-chart}
\end{table}

\section{Method}

The input to our method is a sequence of decimations of the original mesh which match its topology, but, in general, overlap. The output is a new embedding for each mesh (i.e. new vertex positions) such that each tightly contains all finer meshes. Our approach works pairwise, from the finest layer to coarsest. Each inductively contains all finer layers.

The pairwise problem is broken down into two stages: \emph{flow} the fine mesh inside the coarse mesh  (Figure~\ref{fig:swat}) and then \emph{re-inflate} the fine mesh to its original position while pushing the coarse mesh outward.\\
\\
\textbf{Flow:} We first find new positions $\overline{\mathbf{F}}$ for the fine
mesh that minimize total signed distance to $\hat{C}$ integrated over all
deforming surface points $\overline{\textbf{p}} \in F$
\begin{equation}
\Phi(\overline{\mathbf{F}}) = \int_F s(\overline{\mathbf{p}}) d(\overline{\mathbf{p}}) dA,
\label{eq:flow_energy}
\end{equation}
where $d(\overline{\mathbf{p}})$ is the unsigned distance from $\mathbf{p}$ to the 
coarse mesh and $d(\overline{\mathbf{p}})$ modulates by the appropriate sign
($1$ outside $-1$ inside).

%\begin{figure*}[t]
 % \includegraphics[width=\linewidth]{figs/rampant-flow}
 % \caption{Write caption}
 % \label{fig:rampant-flow}
% \end{figure*}

\begin{figure*}[t]
  \includegraphics[width=\linewidth]{figs/zoo_mini}
  \caption{We show robustness of our method by generating 50 nested cages around \emph{MaxPlank}'s head (left). For the \emph{Octopus} mesh we have generated 7 volume minimizing cages and for the higher-genus \emph{Feritility} we have generated 7 layers minimizing surface deformation. }
  \label{fig:minizoo}
\end{figure*}

We minimize this energy by taking small steps opposite the gradient 
direction for each vertex position $\overline{\mathbf{f}}$ in $\overline{\mathbf{F}}(t)$ as 
a function of a fictitious time variable $t$:
\begin{equation}
\frac{\partial \overline{\mathbf{f}}}{\partial t} = - \nabla_{\overline{\mathbf{f}}} \Phi(\overline{\mathbf{F}}).
\label{eq:energy_gradient}
\end{equation}

This flow is non-trivial to discretize. We first approximate the integral with higher-order quadrature points sampled inside fine mesh facets. Then we employ a technique common to surface registration. During each small step of the flow, for each point $\overline{\textbf{p}}$ we freeze its associated closest point on $\hat{C}$, therefore linearizing its signed distance gradient direction. We continue along this flow until the fine mesh is full inside the coarse mesh. In contrast to normal flows, this flow guides the fine mesh toward the medial axis of the coarse mesh. \\
\\
\textbf{Re-inflation:} Next we reverse the flow, marching the fine mesh back along its path, but now we resolve collisions with the coarse mesh \emph{\`{a} la} physically based simulation. There are many feasible embeddings for the coarse mesh so we regularize, treating the problem as a constrained optimization:
\begin{equation}
\min_{\mathbf{C}} E(\mathbf{C}),
\end{equation}
subject to $C$ does not intersect itself or $\overline{F}$.

In physics parlance, we find a static solution for each reverse time step as $\overline{F}$ moves like an infinite mass obstacle toward its original position and pushes the deformable C outward. We leverage state-of-the-art physically-based simulation contact mechanics methods to satisfy these constraints. There are a variety of reasonable choices for $E$: total volume, total displacement, surface deformation, volumetric deformation.

\section{Results}

\begin{figure}[t]
  \includegraphics[width=\linewidth]{figs/swat-cage}
  \caption{Given the a mesh and an initially overlapping coarse mesh (left)
  we flow it until it is completely inside the coarse mesh. We then re-inflate the fine 
  mesh pushing the coarse mesh away to generate an enclosing cage. This cage
  can be used for deformation.}
  \label{fig:swat}
\end{figure}

\begin{figure}[t]
  \includegraphics[width=\linewidth]{figs/octopus-poisson}
  \caption{A single multiresolution v-cycle takes 1.4 secs on this 7M-vertex volumetric Poisson equation. With 14 more iterations (21 secs) the residual error matches a direct solver's (11 mins, back substitution only).}
  \label{fig:octopus}
\end{figure}

In Figure~\ref{fig:minizoo} we show some nested cages produced by our method and in Figure~\ref{fig:swat} we show a single cage result. Our method took some minutes to generate these results. The input \emph{S.W.A.T. man} mesh has a lot of artefacts such as self-intersecting pairs and non-manifold edges and our flow is robust to them. In this case we have obtained a symmetric enclosing cage by minimizing a symmetry energy $E$ over the coarse mesh.

Our cages find immediate application to solve volumetric PDEs using finite elements (Figure~\ref{fig:octopus}). Nestedness makes linear interpolation a good prolongation operator for the v-cycles, and performing multigrid with our meshes compares favourably over direct solvers in terms of speed and memory.

\section{Ongoing work}

The flow in (\ref{eq:energy_gradient}) may fail to flow inside when fine and initial coarse meshes have very different medial axes or when the fine mesh has thin features. We are now considering regularizations of this flow and searching for properties of the initial decimations that provide our method guarantees to work. We also look forward to improving upon our method by considering optimizing all layers simultaneously, instead of pairwise.

%Use of Bibtex is not necessary, but strongly recommended.
%Please make sure that references are displayed in a uniform fashion
\bibliographystyle{abbrv}
\bibliography{references}
 
\end{document}


