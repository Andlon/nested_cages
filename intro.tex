\section{Introduction}
\label{sec:introduction}

As the complexity and size of objects we try to simulate and model continues to
grow, acceleration algorithms for handling these large meshes are becoming
increasingly important. One powerful technique is to decompose the
high-resolution mesh into a hierarchy of coarser approximations or
\emph{cages}: for example, Eulerian PDEs can be solved efficiently on the fine
mesh using any of several successful multiresolution FEM techniques, which
first solve for a coarse approximate solution on the coarsest levels of the
hierarchy, and then polishing that solution on successively finer levels.
Coarse cages also find heavy use in animation, where artist adjustments of the
coarse cage is propagated to the fine model, and in collision detection, where
a coarse cage can be used to efficiently but conservatively exclude pairs of
objects from having collided. For all these applications, the key to good
performance is the ability to generate a good-quality multiresolution
hierarchy.

\noindent \textbf{Desiderata} Although different applications place different
emphases on what makes for a desirable multiresolution hierarchy, an ideal
coarse cage possesses three properties that are particularly important across a
wide spectrum of applications:
\begin{itemize}
\item It completely \emph{encloses} the fine mesh as well as any of the finer
levels of the multiresolution hierarchy. Most algorithms for transferring the
pose of a cage to a detailed object only guarantee small distortion of the
object and lack of artifacts like element inversion if the object is entirely
contained withing the cage, for instances. And if the cage is to be used to
accelerate collision detection, it is essential that the cage conservatively
enclose the object. \todo{Some plausible FEM motivation for this item}
\item It respects the \emph{topology} of the fine mesh and does not introduce
any large global errors in the object geometry. It is important to
differentiate between two types of distances when building the hierarchy:
\emph{embedded distance} in ambient space, and \emph{intrinsic distance}
between two points, the distance of the shortest path through the material
connecting the two points. It is possible for two points that are nearby in
embedded distance to be very far apart in intrinsic distance: consider, for
instance, a model of a human leaning over and nearly touching her toes: the
fingertips and toes are near in ambient space but intrinsically are meters
apart. It is critical for many applications that the coarsening process does
not fuse together these parts of an object. In the case of the human model, it
is unacceptable if trying to animate the model's arm causes the leg to move as
well. As another example, consider calculating the vibration modes of a broken
ring; if the coarsening process fuses the ring together, the modes computed
will be totally incorrect.
\item It fits \emph{tightly} around the next-coarser level of the hierarchy.
Formally, tightness can be quantified using a user-specified energy that
minimizes Hausdorff distance between the cage and the fine mesh, or minimizes
the volume enclosed by the two meshes, etc. A tight fit minimizes the error
incurred by simplifying the input mesh.
\end{itemize}
Existing popular approaches to building a coarse cage satisfy some, but not
all, of these three properties. For instance,
\begin{itemize}
\item Simple decimation of the original surface mesh yields cages that fit very
tightly and preserve a high amount of detail for a given coarsened resolution,
and can preserve the topology of the original mesh, but there is no guarantee
that the decimated mesh will enclose the original.
\item Voxellizing the object to a regular grid is guaranteed to produce an
enclosing cage, but the fit will not be tight and intrinsically distant
features can be fused together if the grid is too coarse. Using an adaptive
grid or octree can improve the fit and topological fidelity, but unless the
object contains mostly axis-aligned features the resulting cage will not
efficiently fit the object.
\item Level-set methods can find a well-fitting cage that encloses the object,
but can fuse spatially proximate features, changing the object topology.
\end{itemize}

\noindent\textbf{Our Contribution} We propose an algorithm for computing a
multiresolution hierarchy of cages that unlike previous approaches is
\emph{guaranteed} to satisfy all three of the desired properties. Given the
mesh $M_i$ for level $i$ of the hierarchy, we compute the next-coarsest level
$M_{i+1}$ using three steps (see figure XXX): first, we perform
topology-preserving decimation to get a coarse, but not enclosing, cage
$C_{i+1}$. We then shrink $M_i$, using a mixture of area- and
distance-minimizing flow (see section XXX) until $M_i$ is completely enclosed
by $C_{i+1}$. Finally, we linearly expand the vertices of $M_i$ back towards
their original positions. We perform this expansion as a physical simulation
(see section XXX), where the motion of $M_i$ is prescribed, contact response
guarantees that $C_{i+1}$ remains an enclosing cage, and a fairness energy
maximizes the tightness of the fit. The final position of $C_{i+1}$ after this
simulation becomes the next cage $M_{i+1}$.


