\section{Method}
\label{sec:method}

% establish input and output
The input to our method is a sequence of $k+1$ potentially overlapping
triangles meshes $(\V_0,\F_0),(\V_1,\F_1),\dots,(\V_k,\F_k)$, where $\V_i$ is a
list of 3D vertex positions and $\F_i$ is a list of triangle index triplets.
%
In general, we only assume that each mesh \emph{approximates} the surface of the same
solid object.
%
In a typical scenario, $(\V_0,\F_0)$ is a high-resolution \emph{original} mesh and
$(\V_1,\F_1),\dots,(\V_k,\F_k)$ are decimations of decreasing resolution.
%
We require that each input mesh is \emph{watertight} \cite{Dey:2003jf}:
%
free of open boundaries, non-manifold elements, or
self-intersections.\footnote{Our input assumptions are stricter than the
\emph{solidness} of \cite{Bernstein:2013:PHH}. That definition permits
non-manifold ``shared'' vertices and edges which confuse decimators.}
%
Depending on the application, troublesome input meshes may be \emph{cleaned} as
a pre-process using available tools (e.g.\
\cite{Attene:2010vv,Jacobson:WN:2013}).

The output of our method is a new sequence of $k$ lists of vertex positions 
$\V'_1,\dots,\V'_k$ such that all points on $(\V'_{i-1},\F_{i-1})$ are
strictly
\emph{inside} $(\V'_i,\F_i)$ for $i=1\dots k$, letting $\V'_0 = \V_0$ (see
\reffig{2d-nested-layers-notaion}).

We opt to alter only the vertex positions (\emph{geometric embedding}) of each
mesh, and not the number or connectivity of vertices ($\F_i$ are unchanged). Among
other benefits, this design decision ensures that the number of vertices in each
mesh is exactly maintained.

The \emph{nesting} property of the output meshes is easily \emph{verified} by testing
that the winding number of at least one vertex of $\V'_{i-1}$ with respect to
$(\V'_i,\F_i)$ is positive and no intersections exist between
$(\V'_{i-1},\F_{i-1})$ and $(\V'_i,\F_i)$. 

Now we describe a general method that guarantees this nesting property while
optimizing any problem specific energy (e.g.\ distance between meshes, mesh
volumes).

Our method operates recursively on two meshes of the sequence at a time: we
compute $\V'_i$ by considering only the solution to the previous level
$(\V'_{i-1},\F_{i-1})$ and initial mesh $(\V_i,\F_i)$. 
%
Without loss of generality, let us simplify our notation and consider computing
$\V'_1$ from $(\V'_0,\F_0)$ and $(\V_1,\F_1)$. We refer to the $0$th
mesh as the ``fine mesh'' and the $1$st mesh as the ``coarse mesh''.
%
Computing new coarse mesh vertex positions $\V'_1$ involves three phases:
\emph{flow} the fine mesh until fully inside the coarse mesh, \emph{re-inflate}
the fine mesh to its original embedding while \emph{pushing} the coarse mesh
out of the way, and finally \emph{optimize} the coarse mesh embedding (see
\reffig{2d-pipeline}).

\subsection{Flow}
\label{sec:flow}

Without loss of generality, the input fine mesh $(\V'_0,\F_0)$ and coarse mesh
$(\V_1,\F_1)$ are free of \emph{self}-intersections,
%
but, in general, the fine mesh will overlap the coarse mesh: some triangles of
the fine mesh will intersect those of the coarse mesh, implying that some
portion of $(\V'_0,\F_0)$ lies outside of $(\V_1,\F_1)$. Equivalently, there
exists a non-empty set of points on $(\V'_0,\F_0)$ with \emph{positive} signed
distance with respect to $(\V_1,\F_1)$ \footnote{Assuming a negative
inside, positive outside convention.}.
%
Our idea is to find a new embedding $\widehat{\V}_0$ that \emph{minimize
signed distance} integrated over all points $\p$ of the fine mesh to the
closest point $\q$ on the coarse mesh:
\begin{align}
\label{equ:sd-energy}
 Φ(\widehat{\V}_0) &= ∫_{(\widehat{\V}_0,\F_0)} s(\p) u(\p) \,dA,\\
& u(\p) = \left\| \p - \argmin_{\q \in (\V_1,\F_1)} \|\p-\q\|\right\|,\\
& s(\p) = \begin{cases}
  1 & \text{ if $\p$ is outside $(\V_1,\F_1)$},\\
  0 & \text{ if $\p$ is exactly on $(\V_1,\F_1)$},\\
  -1 & \text{ if $\p$ is inside $(\V_1,\F_1)$},
\end{cases}
\end{align}

where $u(\p)$ is the \emph{unsigned} distance from $\p$ to the coarse mesh and
$s(\p)$ modulates by the appropriate sign.

\leo{$dA$ has to be fixed, and we are using the one induced by the initial embedding.
If we don't fix, we don't flow all the way towards the medial axis. I have an analytic example
for a sphere, not sure how much details we should provide for this point.}

\leo{I would prefer to use $d$ for distance, instead of $u$}.

\alec{Are we actually using signed \emph{squared} distance?}

\leo{We are using the one whose gradient is the vector connecting to the closest 
point on the coarse mesh. I think it's not squared then.}

It is important that $\p$ consider \emph{all points} on the fine mesh, not just
vertices. Though \emph{all vertices} in $\widehat{\V}_0$ may lie inside
the coarse mesh, parts of edges and facets might still lie outside (see
\reffig{2d-cutting-corner}). It is also important that $\q$ consider \emph{any
point} on the coarse mesh, as the nearest vertex may be arbitrarily farther
away than the closest point along an edge or facet.

We may immediately write our energy as sum of integrals over each triangle 
$\{a,b,c\}$ in $(\widehat{\V}_0,\F_0)$:
\begin{equation}
Φ(\widehat{\V}_0) = ∑_{\forall \{a,b,c\}} ∫_{\p \in \{a,b,c\}} s(\p) u(\p)\,dA.
\end{equation}

We minimize the energy by taking small steps opposite the gradient direction
for each vertex position $\widehat{\vv}$ in $\widehat{\V}_0(t)$ as a function
of a hypothetical \emph{time} variable $t$:
\begin{equation}
\dd{\widehat{\vv}}{t} = -\Grad_{\widehat{\vv}} Φ(\widehat{\V}_0).
\end{equation}
%
By following this gradient, we \emph{flow} the fine mesh vertices, until
$\widehat{\V}_0(t)$ is fully inside the coarse mesh.

Our continuous energy in \refequ{sd-energy} is similar to the data terms found
in non-rigid registration techniques \alec{cite non-rigid ICP paper}. However,
the sign modulator $s(\p)$ makes an important difference. Minimizing unsigned
(positive) distances would encourage points toward the surface of the coarse
mesh. Instead, by encouraging negative distances, points are attracted to the
medial axis \emph{within} the coarse mesh.

To differentiate the 
continuous energy in \refequ{sd-energy}, we
%
first approximate the continuous integral using a discrete set of quadrature 
evaluation points $\p_i$ and corresponding weights $w_i$ for each triangle:
\begin{align}
Φ(\widehat{\V}_0) &\approx ∑_{\forall \{a,b,c\}} ∑_i w_i s(\p_i) u(\p_i),\\
& \p_i = 
\lambda_a \vv_a + 
\lambda_b \vv_b + 
\lambda_c \vv_c,
\end{align}
where $\lambda_a,\lambda_b,\lambda_c$ are the barycentric coordinates locating $\p_i$
in the triangle $\{a,b,c\}$.

\leo{If $w_i$ are per-triangle weights (area in the initial embedding), I think it would be more clear
to put it outside the second summation and replace by $w_{\{a,b,c\}}$ (since it corresponds
to the $\{a,b,c\}$ triangle, not quadrature point $p_i$).}

%
We use second-order quadrature rules and see diminishing returns with more
exact schemes.
%
However, the difficulty of differentiating $u(\p_i)$ remains. To tackle this,
we adapt the successful iterative closest point approach of non-rigid
registration techniques. Namely, we assume that the closest point $\q_i^*$ to each
$\p_i$ remains constant during each small time step. Likewise, we assume that
the sign $s(\p_i)=s_i^*$ remains constant during each time step.

Now, the we can derive a
gradient for the $j$th mesh vertex $\widehat{vv}_i$. 
Without loss of generality, if we assume triangle indices ${a,b,c}$ are always
\emph{rotated} so that $a=j$, if any, then 
\begin{align}
\dd{\widehat{\vv}_i}{t} &\approx -\Grad_{\widehat{\vv}_i} 
∑_{\forall \{a,b,c\} | a = j} ∑_i w_i s(\p_i) u(\p_i),\\
&= -\Grad_{\widehat{\vv}_i} 
∑_{\forall \{a,b,c\} | a = j} ∑_i w_i s_i^* \|\p_i -\q_i^*\|,\\
&= -
∑_{\forall \{a,b,c\} | a = j} ∑_i w_i s_i^* \Grad_{\widehat{\vv}_i} \|\p_i -\q_i^*\|,\\
&=-
∑_{\forall \{a,b,c\} | a = j} ∑_i w_i s_i^* \Grad_{\widehat{\vv}_i}
\|
\lambda_a \vv_i + 
\lambda_b \vv_b + 
\lambda_c \vv_c
-\q_i^*\|,
\end{align}
\alec{whhhhhhaaat. My gradient has run amok. How come I'm not getting that we
have a weighted sum of directions toward the closest points to each quadrature
point?}

\leo{It would be easier to call $s \cdot u$ as a single function, to avoid product rules.
Also, you are forgetting a chain rule (since each quadrature points depends on the embedding).
I have it all written down in a pptx, could share or simply write this derivation.}
\begin{align}
&=-
∑_{\forall \{a,b,c\} | a = j} ∑_i w_i s_i^* \g_i,\\
&\text{ where } \g_i = \begin{cases}
\frac{\p_i -\q_i^*}{\|\p_i -\q_i^*\|} & \text{ if } \|\p_i
-\q_i^*\|<\epsilon,\\
\n(\q_i^*) & \text{ otherwise },
\end{cases}
\end{align}
where $\n(\q_i^*)$ is the unit normal at $\q_i^*$. This normal is chosen with
care. To ensure that it points \emph{inside} the coarse mesh, we determine if
$\q_i^*$ lies near a vertex, near an edge or in the middle of a triangle and
use an interior angle weighted average of incident triangle normals, a uniform
average of incident triangle normals or triangle normal,
respectively \cite{Baerentzen:2005:SDC}.

We iterate this flow stepping with magnitude proportional to a small ``time
step'': $∆t \approx XXXXXX$ \alec{what do we use?} \leo{Default is $10^{-3}$,
which works well for models scaled between -1 and 1. Can be tuned as optional parameter}. 
After each time step we
recompute signs $s_i^*$ and closest points $\q_i^*$ all quadrature points. We
terminate if all signs are negative \emph{and} no intersecting faces are found
between $(\widehat{\V}_0,\F_0)$ and $(\V_1,\F_1)$.

There is no formal, provable guarantee that this flow will succeed. Indeed, in
rare, difficult cases the flow converges without moving the fine mesh fully
inside the coarse mesh. This may occur if the medial axis of the coarse mesh
lies outside of the fine mesh. This is especially rare if the input sequence
results from decimation as decimation effectively \emph{prunes} and simplifies
a mesh's medial axis.

Rather than admit defeat, we propose a additional step which alleviates many
occurrences of this problem. We reverse the picture and expand the coarse mesh,
flowing it away from the current fine mesh along its signed distance field.

\leo{I think it is too early to put this last paragraph here. I'd rather discuss it in a limitation section.
Could be a bit harming here. It would be nicer to show working results here.}

\alec{Might want to answer question: why not just expand the coarse mesh or
always do both?}

\leo{Expansion is more complicated because it is performed in a coarser mesh. Additionally,
we have to run a physical simulation at every step to guarantee it does not tangle, which 
is not needed for the fine mesh.}

\alec{Relationship to David's penetration depth energy: needs intersection free
state; and Untangling Cloths' volume minimization: might push coarse mesh
\emph{inside} fine mesh instead.}

\alec{Should state that we 1) don't care if $(\widehat{\V}_0,\F_0)$ becomes
self-intersection or 2) cannot become self-intersecting.}

\leo{We don't care.}

\subsection{Re-inflation}
\label{sec:reinflation}

At this point, we have flowed the fine mesh vertices so that
$(\widehat{\V}_0,\F_0)$ is fully inside $(\V_1,\F_1)$. We now restore the fine
mesh to its original vertex positions $\V_0$, detecting and resolving collisions with
the coarse mesh along the way.

To do this, we iterate through the vertex positions computed for each time step
during the flow described in \refsec{flow} \emph{in reverse}.
%
Doing so without moving the coarse mesh vertices would immediately result in
fine-coarse intersections.

We can describe each reverse step in our flow in terms of a displacement per
time step, that is, in terms of \emph{velocities}. For the fine mesh, the
positions after the next reverse time step are known, and thus so are its
velocities:
\begin{align}
\widehat{\V}_0(t-∆t) &= 
\widehat{\V}_0(t) - ∆t \widehat{\U}_0(t),\\
\widehat{\U}_0(t) &= 
\frac{\widehat{\V}_0(t) - \widehat{\V}_0(t-∆t)}{∆t}
\end{align}
where $\widehat{\V}_0(t)$ are the vertex positions of the fine mesh at ``flow
time'' $t$ with the positions returning to their input positions at time zero
$\widehat{\V}_0(0) = \V'(0)$, and $\widehat{\U}_0(t)$ are the instantaneous
per-vertex 3D velocity vectors at time $t$.
%
\alec{The signs are funny since we're flowing backwards.}

For the coarse mesh, the positions $\bar{\V}_1(t-∆t)$---and in turn velocities
$\U_1(t)$---are not fixed. In general, there are an infinite number of
\emph{feasible} choices of $\U_1(t)$ so that the repositioned coarse mesh
$(\bar{\V}_1(t-∆t),\F_1)$ remains free of intersections with itself and with the
\emph{re-inflating} fine mesh $(\widehat{V}_0(t-∆t),\F_0)$.
%
To regularize this problem, we minimize the change in position, or equivalently
minimize the magnitude of velocities.
%
Our reverse time step problem becomes:
%
\begin{align}
\label{equ:simulation}
&\argmin_{\U_1(t)} \left\|\U_1(t)\right\|^2,\\
&\text{ subject to: }\\
&(\bar{\V}_1(t-∆t),\F_1) \text{ does not intersect itself},\\
&(\bar{\V}_1(t-∆t),\F_1) \text{ does not intersect } (\V_0(t-∆t),\F_0).
\end{align}

\alec{This is just disallowing instantaneous collisions. We actually model a
harder problem: no continuous collisions. ``Why?''}

By reformulating our problem in a manner familiar to physically based
simulation, we may leverage state of the art contact detection and response
methods.
%
Abstractly, we can treat these methods as a ``black box'', providing it
the fine mesh $(\V_0(t),\F_0)$ and coarse mesh $(\bar{\V}_1(t),\F_1)$ at the end of the
reverse flow time step and the desired velocities $\U_0(t)$ and $\U_1(t)$. 


There remains one interesting twist. Our problem requires the fine mesh to
return exactly to its original positions. In physically-based simulation
parlance, this is tantamount to assigning the fine mesh \emph{infinite mass}.
%
Many methods are fundamentally incapable of handling such constraints
\alec{cite synchronous penalty-based methods?}.
%
Instead, we experimented with two methods.

First, we adapt the ``velocity adjuster'', \alec{there's a better word than
``adjuster''} surface tracking method of \cite{Brochu:2009} to deal with
infinite masses. \alec{What did we have to change in El Topo? We decrease the
time step and retry assuming linear motion?}

\leo{We removed a phase called RIZ (Rigid Impact Zones), that allowed infinite mass 
vertices to not reach final positions. We also replaced Cholmod by IGL's min quad with
fixed (it better handled difficult cases). Finally, we also decrease time step as you described.}

In difficult cases, too many time step subdivisions are needed, suggesting
failure to make progress in finite time. To handle these hard cases, we fall
back on a second, more robust but slower method: speculative asynchronous
contact mechanics \cite{Ainsley:2012:SPA}. This method is an extension of the
only known method to guarantee intersection prevention and positive progress
for \emph{finite mass} objects \cite{Harmon:2009}. Our infinite masses pose an
interesting stress test for this method, but we see success, albeit at a slower
pace.

\subsection{Optimization}

So far we have allowed the coarse mesh positions to \emph{drift} as they are
\emph{pushed} outward by the re-inflating fine mesh. Afterwards, the embedding
of the coarse mesh $(\bar{\V}_1(0),\F_1)$ is strictly outside the fine mesh,
but may have strayed from an \emph{optimal} embedding.

Our final phase improves the positioning of the coarse mesh. The formulation is
general and allows the user to define the optimality metric for the
domain-specific problem at hand. 

Given a differentiable energy $E(\V'_1)$ and a current, feasible embedding
$(\V'_1,\F_1)$ we set up the following ``simulation'', reminiscent of
\refequ{simulation}:
\begin{align}
&\argmin_{\U'_1} \left\|\U'_1 - δ \Grad E(\V'_1) \right\|^2,\\
&\text{ subject to: }\\
&(\V'_1+\U'_1,\F_1) \text{ does not intersect itself},\\
&(\V'_1+\U'_1,\F_1) \text{ does not intersect } (\V'_0,\F_0).
\end{align}
where $δ \approx XXXXXX$\alec{what do we use?} \leo{Default is starting with
$δ = 10^{-1}$ and do a binary search to decrease energy while $δ > 10^{-3}$} is a small step size
parameter and $\Grad E(\V'_1)$ is the gradient of the user supplied energy with
respect to vertex positions. We solve this problem iteratively, setting $\V'_1
\leftarrow \V'_1 + \U'_1$ after each step until convergence using the same
black box solvers as in \refsec{reinflation}.

We experimented with a few different choices of optimality measures.
One natural choice is to minimize the coarse layer's volume, preferring a
tighter fit. In this case, we define:
\begin{align}
E_\text{volume}(\V'_1) &= ∫_{(\V'_1,\F_1)} 1 \,dA,\\
\Grad E_\text{volume}(\V'_1) &= \N_1,
\end{align}
where $\N_1$ are the per-vertex, area-weighted normals \alec{Should just be a
citation for this}.

\alec{what other energies? Dirichlet? ARAP? symmetry? planarization? Do we
really need to include definitions and gradients for all of these?}

\leo{Symmetry, planarization and proximity are partially implemented, could be done.
If we are going to have a supplemental PDF for other reasons, I would then add 
derivations there.}

\alec{Need to address: ``why not optimize while reinflating? why optimize only
at the end?'' We prioritize feasibility and performance before
quality...hmmm...that doesn't sound good.}

\leo{Because we are not concerned about the quality of intermediate steps.
We only care about the quality of the cage itself. And indeed it is faster.}

\alec{Need to address: ``why not try avoid collisions during decimation?''}

\alec{Might want to address: ``why accept pre-decimated meshes? Why not include
decimation as part of pipeline''?}

\leo{Default is to decimate while layers are generated. I.e., when we finish one layer, 
this layer is decimated to be the coarse layer for the next round. Alternatively,
all coarse layers may be prescribed as input.}

\alec{Need to address: ``why not shrink wrap a mesh from well outside?''}

\leo{The way we are doing now starts the final minimization at something close to what
we want, so less risk of finding local minima.}

\alec{We could handle energies which measure across layers. Minimize squared
separation distance between adjacent layers: might pay to expand middle layer
to balance space between fine and coarse layers. We can model this as one large
optimization problem, and recommend that one still proceed with a block
coordinate descent freezing all but one layer iteratively.}

\alec{Discuss optimization step computation: gradient descent, Newton's,
lerping with Global minimum. Displacement energies: Hessian is identity so
Newton's is equivalent to G.D. ARAP, Hessian is slow and unstable. Local
Global?}

\alec{justification for using signed-distance field: medial axis of coarse mesh
should be simpler version of medial axis of the fine mesh. high freq details
are culled: 2d didactic figure.}

\alec{might need to add smoothing if decimation doesn't obey proximity}
