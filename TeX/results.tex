
\section{Results and applications}
\label{sec:results}

\subsection{Can try right now:}

Zoo of results: at least 20 models.
  - human (have meshes)
  - animals
  - mechanical
    - microscope
    - coupledown
    - iwires
  - engineering
    - space shuttle
    - airplane
  - medical
    - heart
    - pelvis

Always take timings!

Stress tests
  1) Very large number of layers: 100!
    - xyz dragon
  2) Insanely complex fractal-like shape with very close features and many holes
    - buddha
    - stanford "dragon"
    - tangled porcupine-like shape (ask changxi for spiky ball)

Literally design and 3D-print a matryoshka doll. Admittedly there's an extra
constraint we're not considering which is that the nested halves need to be
removable from each other.
  - six or seven layers on the max planck head, thickness in between layers

Automatic cages for deformation. Symmetry preservation, planar quads
  - "cage transfer": fit dash cage or horse cage to another humanoid (or
    animal)

\subsection{Needs coding:}

Basic component needed: multires poisson solver

Multigrid Poisson equation. Show that naive non-nested meshes will have
poor/no convergence: alligator teeth, etc.
  - generic poisson equation in 2d and in 3d
    - compare to level set decimation
    - compare to qslim
  - geodesics in heat 

Multi-res ARAP deformation/physical elastics

Multires automatic skinning weights (i.e. QP solve). See Baohua Wu's thesis.

Collision detection, penetration depth estimation. Unlike AABB or KDOP, cages
are isometry invariant (although so are OABB).
  - cache distance to fine mesh on coarse mesh, then approximate distance to
    fine mesh by finding closest point on coarse mesh and adding fine mesh
    distance. Why need nested then?
  - build interactive demo: bouncing beach balls around rigidly moving object

Adaptive simulation \cite{Debunne:2001:DRD}

Flow application: shrink one mesh inside another. Use our new flow to put a 
character inside a T-shirt for example. Use generalized winding number to 
determine the ?sign?, shrink character towards inside the T-shirt, 
expand character back to its initial position using physical simulation. 

Dynamic update: how much work is it to deform the fine mesh and have the coarse
meshes follow? No need for the flow since we already have a feasible initial
position. Just ``simulation'' at this point?
  - need animation sequence of the fine mesh: alec send frames

\subsection{Would be nice:}

Conservative design with contacts in mind: if I'm designing a mechanical object
I want to be sure that two parts don't interpenetrate each other. Suppose the
shapes were super high-res, then design could be slow, so swap in our nested
meshes as a sort of Level-of-detail hierarchy. I guess this only makes sense
for the static (or at least rigid) objects in the design.

Multigrid fluid simulation. Topology is important.
  - medical example: e.g. heart, naive multires combined two unrelated veins
  - ask fang.

Bijective mapping (Locally injective mapping of tet mesh conforming to all
layers): maintains nesting.
