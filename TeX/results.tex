
\section{Results and applications}
\label{sec:results}

\subsection{Have meshes for:}
  + armadillo
  + bunny
  + gargo
  + horse
  + Model1 (human)
  + octopus
  + pelvis (subdivide initial to 100K or 50K, more layers)
  + Model3 (with more layers and double-check volume minimization)
  + Model7 (more layers)
  + Model9 (more layers)
  + anchor (needs more layers)
  + coupledown retry QSlim, more layers
  + rampant (small number of layers stress test)

Must:
  + Alec: write to Kenshi about coarse quad mesh for cage defo.
  + Alec: set up nest view to output camera params, meshes, colors for Etienne
  - spend less than one day on setting up Linux machine
  - max planck 3d print layers
  - create figures for current results
  - remove writing/reading to files (debug mode only)
  - on medium size example run the matlab profiler:
    - all the time is spent inside el topo mex (or etienne's exe), maybe in
      intersection tests
    - don't have to run until completion
  - rampant: try CGAL signed distance isosurface: smallest iso-value such that
    resolution is close to ours and mesh is fully outside input
    Using `signed\_distance\_isosurface`:
    - 'AngleBound',28 
    - 'SignedDistanceType','pseudonormal'
    - play with 'RadiusBound' and 'DistanceBound' (set to same value) to set the
      resolution (start with something like 0.2)
    - 'Level',as small as possible (positive)
    - Try to create the same layers as the rampart: check if *\_2 intersects the
      *\_1 and *\_1 intersects original (if so then increase level)
  
    while intersecting
      binary search on 'Level'
      while resolution not close to ours
        binary search on 'RadiusBound' and 'DistanceBound'
      end
    end
  - Model4
    - If Mesh from el topo is not well separated, try to separate using a few
      iterations of inflation, then adjust the outer and inner params to run
      Etienne's code.
    - Isosurface for the same number of \_levels\_ minimal number of facets per
      level (probably always greater than ours). Similar binary search this
      time the primary loop over the Angle and Distance bound. The point is
      that CGAL isosurface will have to use very fine meshes to fit in small
      space between knees.
  - couplingdown compare CGAL (done) to OpenFlipper, try to match face count per
    layer
  - create timings table
  - cartoon heart 
  - left atrium
  - large number layers (100), rampant
  - write the paper
  - didactic 2D figures:
    - flow
    - expansion
  - Alec: method
  - Etienne: intro
  - Leo: pass on method
  - Alec: related
  - Etienne: pass on related, method
  - Alec: pass on intro

Should:
  - failure case for simpler flows: mean curvature flow on fine mesh does not
    go inside coarse mesh, may even intersect itself
  - large number of layers on high-res mesh xyz dragon
  - high genus: buddha
  - Automatic cages for deformation. Symmetry preservation, planar quads
    - "cage transfer": fit dash cage or horse cage to another humanoid (or
      animal)
  - Compare running Flow on very nicely tessellated model and poorly
  tessellated and "noisy" model

Nice:
  - Model6 (more layers, better decimation)
  - stanford "dragon"
  - tangled porcupine-like shape (+ask changxi for spiky ball)

- draft of paper

- pelvis (one or two more layers)



Coding (Alec):
 - Basic component needed: multires poisson solver in 3D

 - cage deformation example (with symmetry result)

 - Multigrid Poisson equation. Show that naive non-nested meshes will have
   poor/no convergence: alligator teeth, etc.
     - generic poisson equation in 2d and in 3d
       - compare to level set decimation
       - compare to qslim
     - geodesics in heat 

 - Collision detection, penetration depth estimation. Unlike AABB or KDOP, cages
   are isometry invariant (although so are OABB).
     - cache distance to fine mesh on coarse mesh, then approximate distance to
       fine mesh by finding closest point on coarse mesh and adding fine mesh
       distance. Why need nested then?
     - build interactive demo: bouncing beach balls around rigidly moving object
   
 - Multires automatic skinning weights (i.e. QP solve). See Baohua Wu's thesis.
   
 - Multi-res ARAP deformation/physical elastics

 - Bijective mapping (Locally injective mapping of tet mesh conforming to all
   layers): maintains nesting.

 - Flow application: shrink one mesh inside another. Use our new flow to put a 
  character inside a T-shirt for example. Use generalized winding number to 
  determine the ?sign?, shrink character towards inside the T-shirt, 
  expand character back to its initial position using physical simulation. 
    - Armour: Model4 over human: Model3

Adaptive simulation \cite{Debunne:2001:DRD}

Dynamic update: how much work is it to deform the fine mesh and have the coarse
meshes follow? No need for the flow since we already have a feasible initial
position. Just ``simulation'' at this point?
  - need animation sequence of the fine mesh: alec send frames


Retry horse example with ICP energy instead of volume (to try to reduce
"pancakes")

iWires Alien: try expansion 

neuron negative space: subdivide?

\subsection{Would be nice:}

Conservative design with contacts in mind: if I'm designing a mechanical object
I want to be sure that two parts don't interpenetrate each other. Suppose the
shapes were super high-res, then design could be slow, so swap in our nested
meshes as a sort of Level-of-detail hierarchy. I guess this only makes sense
for the static (or at least rigid) objects in the design.

Multigrid fluid simulation. Topology is important.
  - medical example: e.g. heart, naive multires combined two unrelated veins
  - ask fang.

More along the lines of:
  - engineering
    - space shuttle
    - airplane
  - animals
  - mechanical
    - microscope
    - iwires
