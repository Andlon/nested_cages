\section{Related methods and special cases}

Although our method is the first to solve the nested cage problem for the large
variety of input meshes shown in \refsec{results} \emph{while} interoperating
with arbitrary problem-specific decimation algorithms and fitness functions,
many related approaches have been proposed that work for special cases, or
relax requirements imposed above.

If the algorithm is free to generate the coarse cage $\Cinp$, rather than
accepting it as input from the user, the problem becomes substantially easier,
and has been extensively studied:
%
\begin{figure}[b]
\includegraphics[width=\linewidth]{figs/warrior-poisson-vs-cgal}
\caption{Contouring requires aggressive spacing between distance-field
isolevels to produce valid nesting. Semantically distant parts \emph{fuse}
together, destroying shape-awareness, visualized by a pseudocoloring of a
Poisson solution computed on each level.} \label{fig:warrior-poisson}
\end{figure}

\paragraph{Bounding polytopes.}
%
Most naively, one can simply take $\Cinp$ to be a canonical bounding polytope
(e.g.\ box, KDOP, convex hull). An extension would be to ``shrink-wrap'' the
bounding polytope around $\Finp$ to minimize $E$. This idea has been explored
before \cite{Peterhans:2012,Wang:2013:HPE}, but no full solution has been
developed.
%
This is presumably because---while this approach works well for convex
$\Finp$---it will not give a good fit for meshes with nooks concavities. It is
also not clear how to find a bounding polytope for meshes of nontrivial
topology.
%
Another approach might be to stitch overlapping convex volumes
\cite{Xian:2012tv} using mesh boolean operations, but cage topology and
\emph{quality} become difficult to control without sacrificing control of
resolution.

\paragraph{Offset surfaces for simulation meshes.}
%
Representing $\Finp$ as an implicit surface and \emph{contouring level
sets} is a popular method of creating offset surfaces that nest $\Finp$,
particularly in the context of simulations that use a ``simulation mesh'' as a
proxy for very fine rendered geometry~\cite{Campen:2010}. The challenge is not
just defining a scalar function with an isosurface enveloping the object, but
also contouring it with a piecewise-linear triangle mesh without (i) cutting
corners too harshly and intersecting $\Finp$ or (ii) resorting to fine
resolution, nullifying any performance gains of using the cage.
%
Xu et al~\shortcite{Xu:2014:SDF} define a robust signed distance field, but
pass the field to an off-the-shelf contour mesher without any attempt to
guarantee non-intersection with the input.  Similarly, Shen at
al~\shortcite{Shen:2004:IAI} iteratively refine a moving least squares
iso-surface to enclose an existing model tightly via a global scaling
parameter.
%
Some applications, such as collision detection, can make use of implicit
representations without intermediary meshes,
%
but when a mesh is needed it is not enough that the \emph{continuous}
isosurface does not intersect the input model since generic contouring will
invalidate this.
%
A large enough iso-value or
small-enough resolution tolerance must always be chosen to ensure
non-intersection after contouring. For large iso-values, topological control is
lost and close features are quickly merged (see \reffig{warrior-poisson}),
violating the nestedness of the cage. A similar
approach~\cite{Ben-Chen:2009:SDT} has been used for building deformation cages
around input shapes, where an offset surface is created via Poisson surface
reconstruction~\cite{PoissonSurfaceReconstruction06} (we compare against this
approach in ~\reffig{dane-vs-ben-chen}). \newhl{While the tightness of these cages could be controlled with post-hoc shrinking of the cage, the method can introduce topology changes that are not easily remedied.}
%
For meshes with well defined feature curves, it may be possible to merge coarse
on-surface triangulations of patches \cite{Xian:2013}, though strict nesting is
not sought or guaranteed.

\begin{figure}
  \includegraphics[width=\linewidth]{figs/dane-horse-vs-ben-chen-vs-sander}
  \caption{Methods like \protect\cite{Ben-Chen:2009:SDT} use coarse cages as
  computational workhorses to reduce complexity. Their iterative offset
  heuristic \newhl{does not allow control of the cage topology.}
% oversmoothed, loosely fitting cages.
  %
  Meanwhile, \protect\cite{Sander:2000:SC} attempt to
  maintain nestedness during greedy decimation with local constraints. Cages
  \newhl{are loosely-fitting}, but more importantly self-intersections (orange) and
  intersections with the input model (yellow) eventually accumulate.}
  \label{fig:dane-vs-ben-chen}
\end{figure}

\paragraph{Progressive decimation.} of $\Finp$ using edge collapses, taking care
to place new vertices on the exterior of the current volume by solving a system
of inequality constraints~\cite{Sander:2000:SC}, has found some success in
real-time rendering and collision detection \cite{Platis:CGF}.  Although
the \emph{vertices} of these ``progressive hulls'' are guaranteed to lie
outside of $\Finp$, edges and faces of $\Cinp$ might still
intersect $\Finp$ (and $\Cinp$ might globally self-intersect); see
\reffig{dane-vs-ben-chen}. We tested this method on the entire ``zoo'' of
examples shown in \reffig{zoo} and the supplementary material, and of the 26
examples there, in only one case was the entire hierarchy free of such
intersections. We also observe that for coarse cages these hulls tend to be
more loose-fitting than cages produced by methods that optimize the cage shape
globally (see \reffig{dane-vs-ben-chen}).  Self-intersections resulting from
edge collapses can be corrected with post-hoc mesh repair~\cite{Deng:2011vr},
but this technique relies on temperamental 3D tetrahedral meshing in tight
regions near overlaps and does not consider face-/edge-intersections with
the input model.

\paragraph{Voxelization.} of $\Finp$ will create a nested cage, provided that
the resolution is chosen fine enough to avoid topological
artifacts (as in the case of contouring implicit functions). Naive voxelization
yields dense, inefficient cages \cite{Mehra:2009:AMS}, possibly improved by
progressive decimation~\cite{Xian:2009} or mesh booleans~\cite{Xian:2015}
though also inheriting their respective drawbacks.

\paragraph{Mesh untangling.}
%
All of the above methods require relinquishing some degree of control over
decimation to the nested cage algorithm; algorithms that accept an arbitrary
pre-decimated cage $\Cinp$ are less well-developed. \emph{History-free} cloth
collision response methods (e.g.\
\cite{Baraff:2003:UC,Volino:2006:RSC,Wicke:2006,Ye:2012:ICM}) could be used to
separate $\Cinp$ from $\Finp$ if intersections are not too severe; otherwise
such methods quickly get stuck in local minima.

\paragraph{\newhl{Outward Flow.}}
Before settling on an inward flow of $\Finp$, we experimented with an outward
flow of $\Cinp$ away from $\Finp$, along its signed distance field.
%
For convex meshes this works well\newhl{. However, extensive testing revealed this approach suffers from several difficulties on more complex geometries: 1) Flowing vertices along the signed distance field of $\Finp$ is not guaranteed to resolve all face-face collisions between the fine and coarse meshes; the flow is most robust when the flowing elements are small and the signed distance field does not have too many fine features. Flowing the coarse mesh outward is more fragile than flowing the fine mesh in as both of these factors are less favorable; 2) when flowing the fine mesh inside the coarse mesh we do not need to handle self-collisions within the fine mesh, whereas if inflating the coarse mesh, we do. Adding collision constraints aggravates the fragility of the outward flow; the constraints can prevent a flowing face from ever leaving the volume of the fine mesh (see Figure~\ref{fig:knight-exp-vs-sh}).}

\paragraph{Multiresolution.}
%
Methods based on regular grids or lattices naturally support multiresolution
numerical methods.
%
McAdams et al.\ \shortcite{McAdams:2011}
successfully employ multigrid on voxelized shapes to animate
volumetric elastic characters.
%
The main downside of grid-based methods is their traditionally poor handling of
irregular boundaries.
%
Specialized multigrid methods using adaptive octree to handle complex
boundaries for fractures \cite{Dick:2011} and fluids \cite{Ferstl:2014} exist,
but are not immune to troubles of regular grids: boundaries must be 
represented at fine grid levels to avoid aliasing and retain the input's
topology.
%
To avoid \emph{fusing} together geodesically distant parts, the
shape must be modeled in an accommodating pose. If remodeling is not
possible, the grid size must be chosen smaller than not just the smallest
features but also the \emph{smallest void} between features.
%
One solution is to replicate cells in close areas and manage
adjacencies when two or more replicated patches merge and split
\cite{Teran:2005:CSS,Nesme:2009:PTE,Sykora09}.
%
Cell replication is not only difficult to realize robustly, but also riddles
multigrid numerical methods with expensive, SIMD-breaking boundary handling
code: detracting from the performance gains of memory-efficient regular grids
\cite{Demmel04}.

Multiresolution on unstructured grids or tetrahedral meshes is more
temperamental than regular grids and constructing each level requires care
\cite{fish1995efficient}.
%
Geometric multiresolution schemes typically coarsen an input tetrahedral mesh
by removing all but a subset of its vertices, attempting to connect them in a
reasonable way \cite{guillard1993,Adams:1999:PMS}. Special care is required to
maintain any semblance of the original boundary \cite{Brune:2011}, essentially
devolving into constrained Delaunay tessellation with no guarantee that the
coarsening will not \emph{eat away} large portions of the domain.
%
In many scenarios, the boundary of the domain is assumed to be only as
irregular as the coarsest layer, simplifying level design \cite{feng1997non}.

%\cite{fish1995efficient} just an example of a (nonlinear) multigrid solver
%which expects the user to provide (nested?) tetrahedral meshes as input.
%
%\cite{feng1997non} more complicated restriction operator for Galerkin multigrid
%on non-nested meshes, though it seems it's expected that the surfaces of each
%mesh coincide: user provides meshes. I'm not sure how much better this
%restriction would really be compared to simple barycentric one.

A second group of methods generates a multiresolution hierarchy using
decimation. Unmodified mesh decimations \cite{Garland:1997:SSU} have been used
for adaptive simulations for visco-elastic solids \cite{Debunne:2001:DRD}.
%
Such non-nested cages require extrapolation or one-to-many mappings to
\emph{prolongate} solutions on the coarse levels to finer levels.
%
We show cases where this fails to converge for common linear systems on
irregular domains, and we show better convergence for strictly nesting cages
where prolongation is a purely linear interpolation.

Algebraic alternatives to geometric remeshing also exist
\cite{ruge1987algebraic},
but are recommended only when geometric information is not available
\cite{falgout06}.
%
Recently, \cite{Krishnan:2013:EPL} proposed a
multiresolution preconditioner for Laplace-based systems on images and
surfaces.
%
This method combines the elegance of algebraic techniques with some geometric
information derived from the characteristics of the Laplacian matrix, but is
limited to special problems.


%Alternatives to geometric multigrid: algebraic multi-grid (cite Brandt) or
%algebraic techniques for special problem \cite{Krishnan:2013:EPL} (doesn't
%generalize beyond ``M-matrices'')

%\cite{CGCDP:2002} assume a subdivision lattice containing the deforming shape
%as part of the input. Presumably this cage is modeled manually.

\paragraph{Simplification.}
%
The majority of surface mesh decimation techniques aim to preserve outward
appearance with lower and lower mesh resolution
\cite{Hoppe:1996:PM,Garland:1997:SSU,Melax98}.
%
Along these lines, \cite{gumhold2003intersection} output
decimations free of \emph{self-}intersections, ensuring for example that a
character's clothing stays outside its body.
%
In contrast, our method assumes self-intersection free input decimations and
transforms them into nested layers, with no intersections \emph{across}
hierarchy layers.
%
Recent work has considered more elaborate decimation goals than appearance such
as preserving haptic sensations \cite{Otaduy:2003:SPS}.
%
\cite{Otaduy:2003:CDH} also demonstrate how to combine mesh decimations and bounding
volume hierarchies to achieve faster collision detection and handle without
affecting visual appearance.
%
Our nested cages are hierarchical decimations strictly containing the input
shape, ensuring strictly conservative collision detection.
%
Rather than work against the sophistication of existing shape decimation
techniques, we complement them. Our method arbitrary decimations as input and
nests them as a post process.

\paragraph{Interference-aware processing} We credit
\cite{ContactAwareModeling:2011} for their ground-breaking introduction of
contact handling to mainstream geometry processing. Their work inspires us to
consider the contact and collisions tool-set familiar to physically based
simulation in our geometry processing task.
%
In this task, our novel flow is essential for finding a feasible starting
state.
