\section{Related work}
\label{sec:related}


\paragraph{Multiresolution}
%
Methods based on regular grids or lattices naturally support multiresolution numerical methods.
%
McAdams et al.\ successfully employ multgrid on voxelized shapes to animate
elastic charaters \cite{McAdams:2011}.
%
The main downside of grid based methods is their handling of irregular
boundaries.
%
To avoid \emph{gluing} together geodesically distant parts of the shape, the
shape must be modeling in an accommodating pose. If changing the input is not
possible, then the grid size must be chosen smaller that not just the smallest
features but also the \emph{smallest void} between features.
%
A hypothetical solution is to duplicate cells in close areas and managing
adjacencies when two or more duplicated patches merge and split.
%
This succeeds in the less complicated topologies of 2D cartoons
\cite{Sykora09}, but is avoided in 3D: Wojtan et al. 2011 opt for dynamic mesh
\emph{surgery} to avoid the need for duplication \cite{wojtan2011liquid}.
%
Cell duplication is not only difficult to realize robustly, but also riddles
multigrid numerical methods with expensive, SIMD-breaking boundary handling
code: detracting from the performance gains of memory-efficient regular grids.

Multiresolution on unstructured grids or tetrahedral meshes is more
temperamental than regular grids and constructing each level requires care
\cite{fish1995efficient}.
%
Geometric multiresolution schemes typically coarsen an input tetrahedral mesh
by removing all but a subset of its vertices and attempting to connect them in
a reasonable way \cite{guillard1993,Adams:1999:PMS} create each coarser
tetrahedral. Special care is required to maintain any semblance of the original
boundary \cite{Brune:2011}, essentially devolving into constrained Delaunay
tessellation with no guarantee that the coarsening will not \emph{eat away}
large portions of the domain.
%
In many scenarios, the boundary of the domain is assumed to be only as
irregular as the coarsest layer, simplifying level design \cite{feng1997non}.

%\cite{fish1995efficient} just an example of a (nonlinear) multigrid solver
%which expects the user to provide (nested?) tetrahedral meshes as input.
%
%\cite{feng1997non} more complicated restriction operator for Galerkin multigrid
%on non-nested meshes, though it seems it's expected that the surfaces of each
%mesh coincide: user provides meshes. I'm not sure how much better this
%restriction would really be compared to simple barycentric one.

A second group of methods generate a multiresolution hierarchy using decimation. Unmodified mesh decimations \cite{Garland:1997:SSU} have been used for adaptive
simulations for visco-elastic solids \cite{Debunne:2001:DRD}.
%
Such non-nested cages require extrapolation or one-to-many mappings to
\emph{prolongate} solutions on the coarse levels to finer levels.
%
We show cases where this fails to converge for common linear systems on
irregular domains and better convergence for strictly nesting cages where
prolongation is a purely linear interpolation.

Algebraic alternatives to geometric remeshing also exist,
but are recommended only when geometric information is not available
\cite{falgout06}.
%
Recently, Krishnan et al. \shortcite{Krishnan:2013:EPL} proposed a
multiresolution preconditioner for Laplace-based systems on images and
surfaces.
%
This method combines the elegance of algebraic techniques with some geometric
information derived from the characteristics of the Laplacian matrix, but is
limited to special problems.

\paragraph{Offset surfaces for simulation meshes}
In addition to multiresolution layers, our method is suitable for creating a
single-layer ``simulation mesh'' for physically based animation.
%
A simulation mesh reduces computational complexity and ensures robust
deformations for the complex, artifact-ridden geometries embedded inside them.
%
Previous practitioners have experimented with automatic means to create the
surface of these simulation meshes via contouring offset surfaces
\cite{Campen:2010}.
%
The challenge is not just defining a scalar function with an isosurface
enveloping the object, but also \emph{contouring it} with a piecewise-linear
triangle mesh without 1) cutting corners too harshly and intersecting the input
or 2) resorting to fine resolution nullifying any performance gains.
%
Xu et al. \shortcite{Xu:2014:SDF} define a robust signed distance field, but
pass the field to an off-the-shelf contour mesher without any attempt to
guarantee non-intersection with the input.
%
Similarly, Shen et al.\ \shortcite{Shen:2004:IAI} iteratively refine a moving
least squares iso-surface to enclose an existing model tightly via a global
scaling parameter. Like the signed distance function, though the continuous
iso-surface may not intersect the input model, this is invalidating during
generic contouring.
%
A large enough iso-value or small-enough resolution tolerance must always be
chosen to ensure non-intersection after contouring.
%
For large iso-values, topological control is lost and close features are
quickly merged (see \reffig{warrior-poisson}).
%
Whereas high-resolution contouring could actually lead to a more densely sampled
surface than the input.
%
By deforming input decimations, we constrain our pipeline to ensure its output
will match the desired mesh resolutions. 

\begin{figure}
  \includegraphics[width=\linewidth]{figs/warrior-poisson-vs-cgal}
  \caption{Contouring a shape's distance field requires aggressive spacing
  between isolevels to produce valid nesting. Semantically distant parts
  \emph{fuse} together, destroying the shape-awareness of the volumetric
  Laplacian.} 
  \label{fig:warrior-poisson}
\end{figure}

\paragraph{Automatic cages for deformation}
%
Most generalized barycentric coordinates used for shape deformation expect an
input ``cage'' to be completely exterior to the input shape (e.g.\
\cite{Ju:2005:MVC,HarmonicCoodinates07}).
%
Creating such cages manually is time consuming and breaks automatic pipelines.
%
Ben-Chen et al.\ \shortcite{Ben-Chen:2009:SDT} create exterior cages
automatically by iteratively creating an offset surface via Poisson surface
reconstruction \cite{PoissonSurfaceReconstruction06}. There is no guarantee
that this process will converge. Further, a single-layer cage created with our
method is much tighter fitting for the same vertex count (see
\reffig{dane-vs-ben-chen}).
%
Deng et al.\ attempt to keep coarse meshes exterior during decimation and
resort to post-hoc global intersection resolution to repair invalid output
cages \shortcite{Deng:2011vr}.
%
Xian et al.\ construct automatic cages via simplifying dense voxelization of
the input shape \shortcite{Xian:2009}, stitching oriented bounding boxes
around vertex clusters \shortcite{Xian:2012tv}, or by offsetting coarse
on-surface triangulations of feature curves \shortcite{Xian:2013}.
%
These cages are often dense or loose fitting and may still self-intersect or
intersect the input shape globally, as only local checks are conducted.

\begin{figure}
  \includegraphics[width=\linewidth]{figs/dane-horse-vs-ben-chen}
  \caption{Methods like \protect\cite{Ben-Chen:2009:SDT} use coarse cages a
  computational workhorses to reduce complexity. Their iterative offset
  heuristic produces oversmoothed, loosely fitting cages.}
  \label{fig:dane-vs-ben-chen}
\end{figure}

%Alternatives to geometric multigrid: algebraic multi-grid (cite Brandt) or
%algebraic techniques for special problem \cite{Krishnan:2013:EPL} (doesn't
%generalize beyond ``M-matrices'')

%\cite{CGCDP:2002} assume a subdivision lattice containing the deforming shape
%as part of the input. Presumably this cage is modeled manually.

\paragraph{Decimation}
%
The majority of work on surface mesh decimation aims to preserve outward
appearance with lower and lower mesh resolution
\cite{Hoppe:1996:PM,Garland:1997:SSU,Melax98}.
%
Along these lines, Gumhold et al.\ output decimations free of
\emph{self-}intersections, ensuring for example that a character's clothing
stays outside its body after decimation.
%
In contrast, our method assumes self-intersection free input decimations and
transforms them into nested layers, with no intersections \emph{across}
hierarchy layers.
%
Recent work has considered more elaborate decimation goals than appearance such
as preserving haptic sensations \cite{Otaduy:2003:SPS}.
%
Otaduy et al.\ also demonstrate how to combine mesh decimations and bounding
volume hierarchies to achieve faster collision detection and handle without
affecting visual appearance \shortcite{Otaduy:2003:CDH}.
%
Our nested cages could be used to create hierarchical decimations strictly
containing the input shape, ensuring strictly conservative collision detection.
%
Rather than work against the staggering quality of existing shape decimation
techniques, we complement them. Our method takes as input an arbitrary
decimation and works as a post process to nest each layer.

%Flows
%\cite{Kazhdan2012} remove singularities, also self-intersections
%\cite{Sacht:SIV:2013}, in general, will not move inside original surface

% Alternating global scaling and attraction, attempts to move inside. No
% guarantees, requires parameter tuning \cite{Wang:2008}

%\cite{Tagliasacchi:2012:MCS} coerces flow toward the medial axis

\alec{\cite{ContactAwareModeling:2011} introduces contact handling to geometric
modeling, and inspired us to consider the contanct and collisions toolset familiar
to physically based simulation to our geometry processing task.}

\alec{Relationship to David's penetration depth energy: needs intersection free
state; and Untangling Cloths' volume minimization: might push coarse mesh
\emph{inside} fine mesh instead.}

%\alec{Need to address: ``why not try avoid collisions during decimation?''}
