
% Nice table
\usepackage[usenames,dvipsnames,table]{xcolor}
\usepackage{tabularx}
\usepackage{booktabs}
\definecolor{lightbluishgrey}{rgb}{0.9,0.91,0.95}
\newcommand{\ra}[1]{\renewcommand{\arraystretch}{#1}}
\newcommand{\NO}{{\color{red}\textsf{X}}}
\newcommand{\YES}{$\bullet$}

% Comments
\usepackage{pdfcomment}
\usepackage[usenames,dvipsnames,table]{xcolor}
\newcommand{\todo}[1]{{\bf\textcolor{red}{TODO: #1}}}
\definecolor{offwhite}{rgb}{1.0, 1.0, 0.7}
\newcommand{\alec}   [1]{\pdfcomment[color=offwhite,author=Alec]{Alec: #1}}
\newcommand{\leo}    [1]{\pdfcomment[color=offwhite,author=Leo]{Leo: #1}}
\newcommand{\etienne}[1]{\pdfcomment[color=offwhite,author=Etienne]{Etienne: #1}}
\newcommand{\cheatvspace}[1]{}
\newcommand{\newhl}[1]{\textcolor[rgb]{0.2,0.8,0.2}{#1}}

\usepackage{layouts}

\usepackage{wrapfig}
%%% To use cached, low resolution figures (for a smaller output pdf)
%%% Run:
%%%   mogrify -path lowresfig/ -compress jpeg -density 150 fig/*.pdf
%%% And uncomment this:
%\newcommand{\fig}{lowresfig}
%\usepackage{background}
%\backgroundsetup{%
%  scale=1,       %% change accordingly
%  angle=0,       %% change accordingly
%  opacity=.2,    %% change accordingly
%  color =red,  %% change accordingly
%  contents={{\fontsize{50}{60}\selectfont LOW RES FIGURES}}
%}
%% And comment this:
\newcommand{\figs}{figs}


\newcommand{\refequ}[1] {Equation~(\ref{equ:#1})}
\newcommand{\reffig}[1] {Figure~\ref{fig:#1}}
\newcommand{\refoutsidefig}[1] {Figure~#1}
\newcommand{\reftab}[1] {Table~\ref{tab:#1}}
\newcommand{\reflem}[1] {Lemma~\ref{lem:#1}}
\newcommand{\refsec}[1] {Section~\ref{sec:#1}}
\newcommand{\refapp}[1] {Appendix~\ref{app:#1}}


%%%%%%%%%%%%%%%%%%%%%%%%%%%%%%%%%%%%%%%%%%%%%%%%%%%%%%%%%%%%%%%%%%%%%%%%%%%%%%%%
% MATH (should come at end of diary.cls)
%%%%%%%%%%%%%%%%%%%%%%%%%%%%%%%%%%%%%%%%%%%%%%%%%%%%%%%%%%%%%%%%%%%%%%%%%%%%%%%%
% General
\let\mat = \mathbf
\usepackage{xfrac}
\usepackage{mathabx}
\newcommand{\onehalf}{\sfrac{1}{2}}
% Need this for cases
\usepackage{amsmath}
\usepackage{amssymb}    % need for things like varnothing
\usepackage{cancel}
%http://tex.stackexchange.com/a/53829/13600
\usepackage[T1]{fontenc}
% Specialized symbols ranging from general to specific
\newcommand*{\dittostraight}{\textquotesingle\textquotesingle}
\let\Lap = \Delta
\let\Grad = \nabla
\newcommand{\R}{\mathbb{R}}
\newcommand{\vc}[1]{\mathbf{#1}}
\newcommand{\C}{\mat{C}}
\newcommand{\transpose}{{\mathsf T}}
% directional derivative
\newcommand{\dd}[2]{\frac{\partial#1}{\partial#2}}
% Defined on/Defined over: x|_{∂Ω}
\newcommand{\defon}[2]{\left.{#1}\right|_{#2}}
% **(RE)DEFINE** all \a to be \vc{a}. In the comments I reveal what I'm
% overwriting. Because of this you're likely to get mysterious errors when
% typesetting foreign words:
%
% ! LaTeX Error: \mathbf allowed only in math mode.
%
% Seems like this is used to create an accent mark \a'
\renewcommand{\a}{\vc{a}}
% Used to put a little bar under a letter
\renewcommand{\b}{\vc{b}}
% Used to spell fa\c{c}ade
\renewcommand{\c}{\vc{c}}
% Used to put a little dot under a letter
\renewcommand{\d}{\vc{d}}
\newcommand{\e}{\vc{e}}
\newcommand{\f}{\vc{f}}
\newcommand{\g}{\vc{g}}
\newcommand{\h}{\vc{h}}
% Used to make Dotless i like in Turkish
\renewcommand{\i}{\vc{i}}
% Used to make Dotless i like in Turkish
\renewcommand{\j}{\vc{j}}
% Used to spell Polish and Native American words
\renewcommand{\k}{\vc{k}}
% Used to draw an l with a stroke (Polish?)
\renewcommand{\l}{\vc{l}}
\newcommand{\m}{\vc{m}}
\newcommand{\n}{\vc{n}}
% Used to place a little
\renewcommand{\o}{\vc{o}}
\newcommand{\p}{\vc{p}}
\newcommand{\q}{\vc{q}}
% Used to put a little "ring" ontop of a letter like \r{a} = \aa --> å
\renewcommand{\r}{\vc{r}}
\newcommand{\s}{\vc{s}}
% Used to put a "tie" (inverted U) over two letters
\renewcommand{\t}{\vc{t}}
% Little u over a letter
\renewcommand{\u}{\vc{u}}
% Used for putting a little upside down hat on Slavic words \v{Z}eljko
%\renewcommand{\v}{\vc{v}}
\newcommand{\vv}{\vc{v}}
\newcommand{\w}{\vc{w}}
\newcommand{\x}{\vc{x}}
\newcommand{\y}{\vc{y}}
\newcommand{\z}{\vc{z}}
\newcommand{\A}{\mat{A}}
\newcommand{\B}{\mat{B}}
%\renewcommand{\C}{\mat{C}}
\newcommand{\D}{\mat{D}}
\newcommand{\E}{\mat{E}}
\newcommand{\F}{\mat{F}}
\newcommand{\G}{\mat{G}}
%\renewcommand{\H}{\mat{H}}
\newcommand{\I}{\mat{I}}
\newcommand{\J}{\mat{J}}
\newcommand{\K}{\mat{K}}
\newcommand{\Kz}{\overline{\mat{K}}}
\renewcommand{\L}{\mat{L}}

\newcommand{\inp}[1]{\widehat{#1}}
\newcommand{\out}[1]{#1}
\newcommand{\flow}[1]{\widebar{#1}}
\newcommand{\M}{\mat{M}}
\newcommand{\Minp}{\inp{M}}
\newcommand{\Mout}{M}
\newcommand{\MVinp}{\inp{\M}}
\newcommand{\MVout}{\M}
\newcommand{\Cinp}{\inp{C}}
\newcommand{\cinp}{\inp{\c}}
\newcommand{\einp}{\inp{\e}}
\newcommand{\Cout}{\out{C}}
\newcommand{\CVinp}{\inp{\C}}
\newcommand{\CVout}{\C}
\newcommand{\Finp}{\inp{F}}
\newcommand{\Fout}{\out{F}}
\newcommand{\FVflow}{\flow{\F}}
\newcommand{\Fflow}{\flow{F}}
\newcommand{\fflow}{\flow{\f}}
\newcommand{\pflow}{\flow{\p}}
\newcommand{\qinp}{\inp{\q}}
\newcommand{\FVinp}{\inp{\F}}
\newcommand{\FVout}{\F}

\newcommand{\N}{\mat{N}}
%\newcommand{\O}{\mat{O}}
\renewcommand{\P}{\mat{P}}
\newcommand{\RR}{\mat{R}}
\renewcommand{\S}{\mat{S}}
\newcommand{\T}{\mat{T}}
\newcommand{\U}{\mat{U}}
\newcommand{\V}{\mat{V}}
\newcommand{\1}{\mat{1}}
\newcommand{\W}{\mat{W}}
\newcommand{\X}{\mat{X}}
\newcommand{\Y}{\mat{Y}}
\newcommand{\Z}{\mat{Z}}
\newcommand{\argmin}{\mathop{\text{argmin }}}
\newcommand{\ii}{\vc{i}}
\newcommand{\jj}{\vc{j}}
\newcommand{\kk}{\vc{k}}
\newcommand{\ft}{\tilde{f}}
\newcommand{\cotaa}{\cot{\alpha_a}}
\newcommand{\cotab}{\cot{\alpha_b}}
\newcommand{\cotac}{\cot{\alpha_c}}
\newcommand{\cotba}{\cot{\beta_a}}
\newcommand{\cotbb}{\cot{\beta_b}}
\newcommand{\cotbd}{\cot{\beta_d}}
\newcommand{\cotaone}{\cot{\alpha_1}}
\newcommand{\cotatwo}{\cot{\alpha_2}}
\newcommand{\cotbone}{\cot{\beta_1}}
\newcommand{\cotbtwo}{\cot{\beta_2}}
% unit normal
\newcommand{\un}{\hat{\vc{n}}}



% scientific notation
% http://tex.stackexchange.com/a/70532/13600
\usepackage{siunitx}
\sisetup{output-exponent-marker=\text{e}, bracket-negative-numbers,
open-bracket={\text{-}}, close-bracket={}}

\usepackage[mathletters]{ucs}
\usepackage[utf8x]{inputenc}
% μ
\usepackage{upgreek}
\usepackage{dblfloatfix}
\newcommand{\microsec}{\upmu{}\text{s}}



%  bold paragraph header without space after
\newcommand{\nospaceparagraph}[1]{{\sffamily\textbf{#1}}}
\renewcommand{\Re}{\operatorname{Re}}



%%%%%%%%%%%%%%%%%%%%%%%%%%%%%%%%%%%%%%%%%%%%%%%%%%%%%%%%%%%%%%%%%%%%%%%%%%%%%%%%
% IMAGES, GRAPHICS, FIGURES AND CAPTIONS
%%%%%%%%%%%%%%%%%%%%%%%%%%%%%%%%%%%%%%%%%%%%%%%%%%%%%%%%%%%%%%%%%%%%%%%%%%%%%%%%
\usepackage{graphicx}
\newcommand{\figures}{figures}
\usepackage{wrapfig}
\newcommand{\wf}[6]{%
  \begin{wrapfigure}[#1]{#2}{#3}%
  \centering%
  \includegraphics[#4]{#5}
  #6%
\end{wrapfigure}}
\newcommand{\ig}[2][]{\includegraphics[#1]{#2}}
%\newcommand{\fig}{2}{%
%\begin{figure}%
%\centering%
%#1%
%\caption{#2}%
%\end{figure}}

%%%%%%%%%%%%%%%%%%%%%%%%%%%%%%%%%%%%%%%%%%%%%%%%%%%%%%%%%%%%%%%%%%%%%%%%%%%%%%%%
% PSEUDOCODE
%%%%%%%%%%%%%%%%%%%%%%%%%%%%%%%%%%%%%%%%%%%%%%%%%%%%%%%%%%%%%%%%%%%%%%%%%%%%%%%%
\usepackage[ruled,noend]{algorithm2e}
  \DontPrintSemicolon
  \SetAlgoLined
  \SetKwInput{KwData}{Inputs} 
  \SetKwInput{KwResult}{Outputs}
  \RestyleAlgo{ruled}
  \SetKwBlock{Repeat}{repeat}{}

 % disable begin-----end thing
 \newcommand\newcolumn{\vfil\penalty-10000 }
 \newcommand{\comment}[1]{\hspace{\fill}\textcolor[rgb]{0.5,0.5,0.5}{\emph{// #1}}}
 \newcommand{\longcomment}[1]{\textcolor[rgb]{0.5,0.5,0.5}{\noindent \emph{/* #1 */}}}
\usepackage{array}

\usepackage{xhfill}% http://ctan.org/pkg/xhfill
%\newcommand{\ditto}[1][.4pt]{\xrfill{#1}~\textquotedbl~\xrfill{#1}}
\newcommand*{\ditto}{---\textquotedbl---}

% http://tex.stackexchange.com/a/94702/13600
\newenvironment{absolutelynopagebreak}
  {\par\nobreak\vfil\penalty0\vfilneg
   \vtop\bgroup}
  {\par\xdef\tpd{\the\prevdepth}\egroup
   \prevdepth=\tpd}

