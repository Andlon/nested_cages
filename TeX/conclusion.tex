\section{Future work \& Conclusion}
\label{sec:conclusion}
%
In future, we plan to optimize the performance of both the signed distance
field flow and the re-inflation steps.
%
Collision detection and handling dominates running time and our prototype
naively recomputes acceleration data-structures rather than updating them while
the meshes change dynamically.

Our insight to break the multi-layer nesting problem into pairwise subproblems
ensures tractability, but in some cases leads to converging at an
``artificial local minimum.'' If a coarse cage collides with itself during
inflation then it may create a pinch that blocks inflation of subsequent coarse
layers (see \reffig{homer}).
%
One solution is to iterate through the fine layers to make sufficient room in
these problem areas, but defining this relaxation direction is not obvious.

\begin{figure}
  \includegraphics[width=\linewidth]{figs/homer-fail}
  \caption{If a expanding coarse mesh collides with itself (green), it creates
  a \emph{pinch} preventing processing of further coarser layers.}
  \label{fig:homer}
\end{figure}

In cases where the input coarse cage begins too far away from the fine mesh,
the signed-distance flow will fail: for example, by flowing vertices of a
triangle into opposite parts of the coarse mesh. 
%
Adding a small amount of smoothing on the flowing fine mesh or expanding the
coarse mesh alleviates some of these problems, but a general solution is
allusive.
%
The correct assignment seems related to correctly matching medial axes of both
meshes. Perhaps this is an avenue of future improvement.

Though we are delighted with the performance of our multiresolution solver, 
we would like to further improve it. We believe there are even more gains to be
made by parameter tuning and experimenting with different coarsening
gradations. We would also like to consider using our meshes to build
multiresolution proconditioners for conjugate gradient solvers.
%
We expect that higher order PDEs with more involved boundary conditions will
receive an even greater benefit from our nested cages.

It would be interesting to analyze formally the convergence of our nested cages
along the lines of \cite{chan1996convergence} who consider the then-available
non-nested hierarchies.


In conclusion, nested cages prove to be a powerful tool in a variety of
applications. 
%
Our signed-distance flow consistently finds initial feasible states for our
constraint-based optimization.
%
By leveraging state-of-the-art collision handling tools from physically based
simulation, we are able to generate cages that in turn enable faster physical
simulations, more-efficient linear system solvers and better real-time
deformation user interfaces.
%
We hope that our algorithm's success encourages more multiresolution volumetric
methods using unstructured meshes in geometry processing, computer graphics,
and beyond.
