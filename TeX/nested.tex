%%% The ``\documentclass'' command has one parameter, based on the kind of
%%% document you are preparing.
%%%
%%% [annual] - Technical paper accepted for presentation at the ACM SIGGRAPH 
%%%   or SIGGRAPH Asia annual conference.
%%% [sponsored] - Short or full-length technical paper accepted for 
%%%   presentation at an event sponsored by ACM SIGGRAPH
%%%   (but not the annual conference Technical Papers program).
%%% [abstract] - A one-page abstract of your accepted content
%%%   (Technical Sketches, Posters, Emerging Technologies, etc.). 
%%%   Content greater than one page in length should use the "[sponsored]"
%%%   parameter.
%%% [preprint] - A preprint version of your final content.
%%% [review] - A technical paper submitted for review. Includes line
%%%   numbers and anonymization of author and affiliation information.

\documentclass[review]{acmsiggraph}

\usepackage{graphicx}
\usepackage[usenames,dvipsnames]{color}
\usepackage{amsmath}    % need for subequations
\usepackage{amssymb}    % need for things like varnothing
\usepackage{dsfont}
\usepackage{mathrsfs}
\usepackage{placeins}
\usepackage{microtype}
\usepackage{wrapfig}
\usepackage{soul}

% fix spacing before \paragraph
\usepackage{titlesec}
\titlespacing{\paragraph}{%
  0pt}{%              left margin
  0.2\baselineskip}{% space before (vertical)
  1em}%

%%% If you are submitting your paper to one of our annual conferences - the 
%%% ACM SIGGRAPH conference held in North America, or the SIGGRAPH Asia 
%%% conference held in Southeast Asia - there are several commands you should 
%%% consider using in the preparation of your document.

%%% 1. ``\TOGonlineID''
%%% When you submit your paper for review, please use the ``\TOGonlineID''
%%% command to include the online ID value assigned to your paper by the
%%% submission management system. Replace '45678' with the value you were
%%% assigned.

\TOGonlineid{0171}

%%% 2. ``\TOGvolume'' and ``\TOGnumber''
%%% If you are preparing a preprint of your accepted paper, and your paper
%%% will be published in an issue of the ACM ``Transactions on Graphics''
%%% journal, replace the ``0'' values in the commands below with the correct
%%% volume and number values for that issue - you'll get them before your
%%% final paper is due.

\TOGvolume{}
\TOGnumber{}

%%% 3. ``TOGarticleDOI''
%%% The ``TOGarticleDOI'' command accepts the DOI information provided to you
%%% during production, and which makes up the URLs which identifies the ACM
%%% article page and direct PDF link in the ACM Digital Library.
%%% Replace ``1111111.2222222'' with the values you are given.

\TOGarticleDOI{1111111.2222222}

%%% 4. ``\TOGprojectURL'', ``\TOGvideoURL'', ``\TOGdataURL'', ``\TOGcodeURL''
%%% If you would like to include links to personal repositories for auxiliary
%%% material related your research contribution, you may use one or more of
%%% these commands to define an appropriate URL. The ``\TOGlinkslist'' command
%%% found just before the first section of your document will add hyperlinked
%%% icons to your document, in addition to hyperlinked icons which point to
%%% the ACM Digital Library article page and the ACM Digital Library-held PDF.

\TOGprojectURL{}
\TOGvideoURL{}
\TOGdataURL{}
\TOGcodeURL{}

%%% Replace ``PAPER TEMPLATE TITLE'' with the title of your paper or abstract.

\title{Nested Cages}

%%% The ``\author{}'' command takes the names and affiliations of each of the
%%% authors of your paper or abstract. The ``\thanks{}'' command takes the
%%% contact information for each author.
%%% For multiple authors, separate each author's information by the ``\and''
%%% command.

\author{}

%%% The ``pdfauthor'' command accepts the authors of the work,
%%% comma-delimited, and adds this information to the PDF metadata.

\pdfauthor{}

%%% Keywords that describe your work. The ``\keywordlist'' command will print
%%% them out.

\keywords{}

%%% The ``\begin{document}'' command is the start of the document.

%%% If you have user-defined macros, you may include them here.
% Comments
\newcommand{\leo}[1]{{\bf\textcolor[rgb]{0.9,0.0,0.0}{Leo: #1}}}
\newcommand{\todo}[1]{{\bf\textcolor{red}{TODO: #1}}}
\newcommand{\alec}[1]{{\bf\textcolor[rgb]{0.2,0.8,0.2}{Alec: #1}}}
\newcommand{\etienne}[1]{{\bf\textcolor[rgb]{0.9,0.5,0.0}{Etienne: #1}}}
\newcommand{\cheatvspace}[1]{}
\newcommand{\newhl}[1]{\textcolor[rgb]{0.2,0.8,0.2}{#1}}

% Nice table
\usepackage[table]{xcolor}
\usepackage{tabularx}
\usepackage{booktabs}
\definecolor{lightbluishgrey}{rgb}{0.9,0.91,0.95}
\newcommand{\ra}[1]{\renewcommand{\arraystretch}{#1}}
\newcommand{\NO}{{\color{red}\textsf{X}}}
\newcommand{\YES}{$\bullet$}
% Comments

\usepackage{layouts}

\usepackage{wrapfig}
%%% To use cached, low resolution figures (for a smaller output pdf)
%%% Run:
%%%   mogrify -path lowresfig/ -compress jpeg -density 150 fig/*.pdf
%%% And uncomment this:
%\newcommand{\fig}{lowresfig}
%\usepackage{background}
%\backgroundsetup{%
%  scale=1,       %% change accordingly
%  angle=0,       %% change accordingly
%  opacity=.2,    %% change accordingly
%  color =red,  %% change accordingly
%  contents={{\fontsize{50}{60}\selectfont LOW RES FIGURES}}
%}
%% And comment this:
\newcommand{\figs}{figs}


\newcommand{\refequ}[1] {Equation~(\ref{equ:#1})}
\newcommand{\reffig}[1] {Figure~\ref{fig:#1}}
\newcommand{\refoutsidefig}[1] {Figure~#1}
\newcommand{\reftab}[1] {Table~\ref{tab:#1}}
\newcommand{\reflem}[1] {Lemma~\ref{lem:#1}}
\newcommand{\refsec}[1] {Section~\ref{sec:#1}}
\newcommand{\refapp}[1] {Appendix~\ref{app:#1}}


%%%%%%%%%%%%%%%%%%%%%%%%%%%%%%%%%%%%%%%%%%%%%%%%%%%%%%%%%%%%%%%%%%%%%%%%%%%%%%%%
% MATH (should come at end of diary.cls)
%%%%%%%%%%%%%%%%%%%%%%%%%%%%%%%%%%%%%%%%%%%%%%%%%%%%%%%%%%%%%%%%%%%%%%%%%%%%%%%%
% General
\let\mat = \mathbf
\usepackage{xfrac}
\usepackage{mathabx}
\newcommand{\onehalf}{\sfrac{1}{2}}
% Need this for cases
\usepackage{amsmath}
\usepackage{amssymb}    % need for things like varnothing
\usepackage{cancel}
%http://tex.stackexchange.com/a/53829/13600
\usepackage[T1]{fontenc}
% Specialized symbols ranging from general to specific
\newcommand*{\dittostraight}{\textquotesingle\textquotesingle}
\let\Lap = \Delta
\let\Grad = \nabla
\newcommand{\R}{\mathbb{R}}
\newcommand{\vc}[1]{\mathbf{#1}}
\newcommand{\C}{\mat{C}}
\newcommand{\transpose}{{\mathsf T}}
% directional derivative
\newcommand{\dd}[2]{\frac{\partial#1}{\partial#2}}
% Defined on/Defined over: x|_{∂Ω}
\newcommand{\defon}[2]{\left.{#1}\right|_{#2}}
% **(RE)DEFINE** all \a to be \vc{a}. In the comments I reveal what I'm
% overwriting. Because of this you're likely to get mysterious errors when
% typesetting foreign words:
%
% ! LaTeX Error: \mathbf allowed only in math mode.
%
% Seems like this is used to create an accent mark \a'
\renewcommand{\a}{\vc{a}}
% Used to put a little bar under a letter
\renewcommand{\b}{\vc{b}}
% Used to spell fa\c{c}ade
\renewcommand{\c}{\vc{c}}
% Used to put a little dot under a letter
\renewcommand{\d}{\vc{d}}
\newcommand{\e}{\vc{e}}
\newcommand{\f}{\vc{f}}
\newcommand{\g}{\vc{g}}
\newcommand{\h}{\vc{h}}
% Used to make Dotless i like in Turkish
\renewcommand{\i}{\vc{i}}
% Used to make Dotless i like in Turkish
\renewcommand{\j}{\vc{j}}
% Used to spell Polish and Native American words
\renewcommand{\k}{\vc{k}}
% Used to draw an l with a stroke (Polish?)
\renewcommand{\l}{\vc{l}}
\newcommand{\m}{\vc{m}}
\newcommand{\n}{\vc{n}}
% Used to place a little
\renewcommand{\o}{\vc{o}}
\newcommand{\p}{\vc{p}}
\newcommand{\q}{\vc{q}}
% Used to put a little "ring" ontop of a letter like \r{a} = \aa --> å
\renewcommand{\r}{\vc{r}}
\newcommand{\s}{\vc{s}}
% Used to put a "tie" (inverted U) over two letters
\renewcommand{\t}{\vc{t}}
% Little u over a letter
\renewcommand{\u}{\vc{u}}
% Used for putting a little upside down hat on Slavic words \v{Z}eljko
%\renewcommand{\v}{\vc{v}}
\newcommand{\vv}{\vc{v}}
\newcommand{\w}{\vc{w}}
\newcommand{\x}{\vc{x}}
\newcommand{\y}{\vc{y}}
\newcommand{\z}{\vc{z}}
\newcommand{\A}{\mat{A}}
\newcommand{\B}{\mat{B}}
%\renewcommand{\C}{\mat{C}}
\newcommand{\D}{\mat{D}}
\newcommand{\E}{\mat{E}}
\newcommand{\F}{\mat{F}}
\newcommand{\G}{\mat{G}}
%\renewcommand{\H}{\mat{H}}
\newcommand{\I}{\mat{I}}
\newcommand{\J}{\mat{J}}
\newcommand{\K}{\mat{K}}
\newcommand{\Kz}{\overline{\mat{K}}}
\renewcommand{\L}{\mat{L}}

\newcommand{\inp}[1]{\widehat{#1}}
\newcommand{\out}[1]{#1}
\newcommand{\flow}[1]{\widebar{#1}}
\newcommand{\M}{\mat{M}}
\newcommand{\Minp}{\inp{M}}
\newcommand{\Mout}{M}
\newcommand{\MVinp}{\inp{\M}}
\newcommand{\MVout}{\M}
\newcommand{\Cinp}{\inp{C}}
\newcommand{\cinp}{\inp{\c}}
\newcommand{\einp}{\inp{\e}}
\newcommand{\Cout}{\out{C}}
\newcommand{\CVinp}{\inp{\C}}
\newcommand{\CVout}{\C}
\newcommand{\Finp}{\inp{F}}
\newcommand{\Fout}{\out{F}}
\newcommand{\FVflow}{\flow{\F}}
\newcommand{\Fflow}{\flow{F}}
\newcommand{\fflow}{\flow{\f}}
\newcommand{\pflow}{\flow{\p}}
\newcommand{\qinp}{\inp{\q}}
\newcommand{\FVinp}{\inp{\F}}
\newcommand{\FVout}{\F}

\newcommand{\N}{\mat{N}}
%\newcommand{\O}{\mat{O}}
\renewcommand{\P}{\mat{P}}
\newcommand{\RR}{\mat{R}}
\renewcommand{\S}{\mat{S}}
\newcommand{\T}{\mat{T}}
\newcommand{\U}{\mat{U}}
\newcommand{\V}{\mat{V}}
\newcommand{\1}{\mat{1}}
\newcommand{\W}{\mat{W}}
\newcommand{\X}{\mat{X}}
\newcommand{\Y}{\mat{Y}}
\newcommand{\Z}{\mat{Z}}
\newcommand{\argmin}{\mathop{\text{argmin }}}
\newcommand{\ii}{\vc{i}}
\newcommand{\jj}{\vc{j}}
\newcommand{\kk}{\vc{k}}
\newcommand{\ft}{\tilde{f}}
\newcommand{\cotaa}{\cot{\alpha_a}}
\newcommand{\cotab}{\cot{\alpha_b}}
\newcommand{\cotac}{\cot{\alpha_c}}
\newcommand{\cotba}{\cot{\beta_a}}
\newcommand{\cotbb}{\cot{\beta_b}}
\newcommand{\cotbd}{\cot{\beta_d}}
\newcommand{\cotaone}{\cot{\alpha_1}}
\newcommand{\cotatwo}{\cot{\alpha_2}}
\newcommand{\cotbone}{\cot{\beta_1}}
\newcommand{\cotbtwo}{\cot{\beta_2}}
% unit normal
\newcommand{\un}{\hat{\vc{n}}}



% scientific notation
% http://tex.stackexchange.com/a/70532/13600
\usepackage{siunitx}
\sisetup{output-exponent-marker=\text{e}, bracket-negative-numbers,
open-bracket={\text{-}}, close-bracket={}}

\usepackage[mathletters]{ucs}
\usepackage[utf8x]{inputenc}
% μ
\usepackage{upgreek}
\usepackage{dblfloatfix}
\newcommand{\microsec}{\upmu{}\text{s}}



%  bold paragraph header without space after
\newcommand{\nospaceparagraph}[1]{{\sffamily\textbf{#1}}}
\renewcommand{\Re}{\operatorname{Re}}



%%%%%%%%%%%%%%%%%%%%%%%%%%%%%%%%%%%%%%%%%%%%%%%%%%%%%%%%%%%%%%%%%%%%%%%%%%%%%%%%
% IMAGES, GRAPHICS, FIGURES AND CAPTIONS
%%%%%%%%%%%%%%%%%%%%%%%%%%%%%%%%%%%%%%%%%%%%%%%%%%%%%%%%%%%%%%%%%%%%%%%%%%%%%%%%
\usepackage{graphicx}
\newcommand{\figures}{figures}
\usepackage{wrapfig}
\newcommand{\wf}[6]{%
  \begin{wrapfigure}[#1]{#2}{#3}%
  \centering%
  \includegraphics[#4]{#5}
  #6%
\end{wrapfigure}}
\newcommand{\ig}[2][]{\includegraphics[#1]{#2}}
%\newcommand{\fig}{2}{%
%\begin{figure}%
%\centering%
%#1%
%\caption{#2}%
%\end{figure}}

%%%%%%%%%%%%%%%%%%%%%%%%%%%%%%%%%%%%%%%%%%%%%%%%%%%%%%%%%%%%%%%%%%%%%%%%%%%%%%%%
% PSEUDOCODE
%%%%%%%%%%%%%%%%%%%%%%%%%%%%%%%%%%%%%%%%%%%%%%%%%%%%%%%%%%%%%%%%%%%%%%%%%%%%%%%%
\usepackage[ruled]{algorithm2e}
  \DontPrintSemicolon
  \SetAlgoLined
  \SetKwInput{KwData}{Inputs} 
  \SetKwInput{KwResult}{Outputs}
  \RestyleAlgo{ruled}
 % disable begin-----end thing
 \newcommand\newcolumn{\vfil\penalty-10000 }
 \newcommand{\comment}[1]{\hspace{\fill}\emph{// #1}}
\usepackage{array}

\usepackage{xhfill}% http://ctan.org/pkg/xhfill
%\newcommand{\ditto}[1][.4pt]{\xrfill{#1}~\textquotedbl~\xrfill{#1}}
\newcommand*{\ditto}{---\textquotedbl---}


% FIX LINE NUMBERS AFTER EQUATION
% http://changilkim.wordpress.com/2013/05/13/siggraph-latex-template-line-number-correction-hack/
\newcommand*\patchAmsMathEnvironmentForLineno[1]{%
\expandafter\let\csname old#1\expandafter\endcsname\csname #1\endcsname
\expandafter\let\csname oldend#1\expandafter\endcsname\csname end#1\endcsname
\renewenvironment{#1}%
{\linenomath\csname old#1\endcsname}%
{\csname oldend#1\endcsname\endlinenomath}}%
\newcommand*\patchBothAmsMathEnvironmentsForLineno[1]{%
\patchAmsMathEnvironmentForLineno{#1}%
\patchAmsMathEnvironmentForLineno{#1*}}%
\AtBeginDocument{%
\patchBothAmsMathEnvironmentsForLineno{equation}%
\patchBothAmsMathEnvironmentsForLineno{align}%
\patchBothAmsMathEnvironmentsForLineno{flalign}%
\patchBothAmsMathEnvironmentsForLineno{alignat}%
\patchBothAmsMathEnvironmentsForLineno{gather}%
\patchBothAmsMathEnvironmentsForLineno{multline}%
\patchBothAmsMathEnvironmentsForLineno{eqnarray}%
}

\begin{document}

%%% A ``teaser'' image appears under the title and affiliation information,
%%% horizontally centered, and above the two columns of text. This is OPTIONAL.
%%% If you choose to have a ``teaser'' image, it needs to be placed between
%%% ``\begin{document}'' and ``\maketitle.''

\teaser{
  \includegraphics[width=\textwidth]{\figs/bunny-shelf-teaser}
  \caption{
Given an input shape (top left on shelf), our method constructs \emph{nested} cages:
each subsequent mesh is coarser than the last and fully encloses it without
intersections. A slice through all layers (left), shows a tightly
encaged \emph{Bunny}.}
  \label{fig:teaser}  
}

%%% The ``\maketitle'' command must appear after ``\begin{document}'' and,
%%% if you have one, after the definition of your ``teaser'' image, and
%%% before the first ``\section'' command.

\maketitle

%%% Your paper's abstract goes in its own section.

\begin{abstract} 

\alec{rewrite to match intro}
%
Many tasks in geometry processing and physical simulation benefit from
multiresolution hierarchies. Ideal coarse domain approximations should satisfy
several desiderata for efficient and accurate multiresolution algorithms: the
coarsened domain should respect the original topology, \emph{enclose} the
original object, yet fit as tightly as possible.
%
Existing techniques such as surface mesh decimation, voxelization, or
contouring distance level sets, violate one or more of these desiderata.
%
We propose a solution that satisfies all three requirements by successively
constructing each next-coarsest level of the hierarchy, using a sequence of
decimation, flow, and contact-aware optimization steps.
%
From coarse to fine, each layer then fully encages the next while retaining a
snug fit and respecting the original surface topology.
%
We show that the method is applicable to a wide variety of shapes of complex
geometry and topology.
%
We demonstrate the effectiveness of our nested cages not only for
multiresolution solvers, but also for conservative collision detection, domain
discretization for elastic simulation, and cage-based geometric modeling. 

\end{abstract}




%%% ACM Computing Review (CR) categories.
%%% See <http://www.acm.org/class/1998/> for details.
%%% The ``\CRcat'' command takes four arguments.

%\begin{CRcatlist}
%  \CRcat{}{}{}{}
%\end{CRcatlist}
%
%%%% The ``\keywordlist'' command prints out the keywords.
%
%\keywordlist

%%% The ``\TOGlinkslist'' command will insert hyperlinked icon(s) to your
%%% paper. This includes, at a minimum, hyperlinked icons to the ACM article
%%% page and the ACM Digital Library-held PDF. If you added URLs to
%%% ``\TOGprojectURL'' or the other, similar commands, they will be added to
%%% the list of icons.
%%% Note: this functionality only works for annual-conference papers.

\TOGlinkslist

%%% The ``\copyrightspace'' command 
%%% Do not remove this command.

\copyrightspace

%%% This is the first section of the body of your paper.

%
%%textwidth: \printinunitsof{in}\prntlen{\textwidth}
%%%textwidth: 7.00137in
%%
%%linewidth: \printinunitsof{in}\prntlen{\linewidth}
%%%linewidth: 3.33461in
%%
%%\makeatletter
%%orig: \f@size pt \f@family 
%%%orig: 9pt ptm
%%\makeatother
%%\rmfamily

\section{Introduction}
\label{sec:introduction}

\begin{figure}[b]
  \includegraphics[width=\linewidth]{figs/horse-25-layers}
  \caption{Our tight nesting property is robust even when the number of cages
  is large: 25 around this \emph{Horse}.}
  \label{fig:horse-25-layers}
\end{figure}

As the complexity and size of computational objects continue to grow,
acceleration algorithms become increasingly important. One powerful technique
is to decompose a high-resolution mesh into a hierarchy of increasingly coarse
approximations or \emph{cages} (see \reffig{teaser}). For example,
multiresolution FEM techniques efficiently solve Eulerian PDEs on very fine
meshes by moving up and down the hierarchy; low frequency residual error
disappears quickly on the coarsest levels while fine levels smooth away high
frequency error. Coarse enclosing cages also find use in physical simulation,
where deformations of the cage are interpolated onto embedded high-resolution
geometries; in interactive animation, where artists specify large-scale
deformations by adjusting the low-dimensional cage; and in collision detection,
where conservative culling reduces computation time. For all these
applications, the key to high performance is the ability to generate a quality
multiresolution hierarchy.

The straightforward approach to building a hierarchy around an object $\Fout$
is to use an application-specific decimation algorithm to build a coarse
approximation $\Cout$ to $\Fout$; $\Cout$ itself can be further coarsened to
build the next level of the hierarchy, etc. (see \reffig{horse-25-layers}).
Unfortunately $\Cout$ will typically intersect $\Fout$, which is often
undesirable: most algorithms for transferring the pose of a deformation cage to
a detailed object only guarantee small distortion of the object and lack of
element-inversion artifacts if the object is entirely contained within the cage
\cite{HarmonicCoodinates07,Ben-Chen:2009:VHM}; strict nesting is essential for
accelerating collision detection \emph{conservatively}; and while nesting is
not a necessary condition for multiresolution convergence
\cite{Chan96overlappingschwarz}, the ability to use simple linear interpolation
for prolongation is known to produce more robust, more efficient, and easier to
implement solvers \cite{chan2000robust,dickopf2010multilevel}. This is
particularly important for enforcing Neumann boundary conditions, common to
simulation and geometry processing \cite{chan1999boundary}.

For these reasons, one would like a cage $\Cout$ that is \emph{nested} around
$\Fout$: that surrounds $\Fout$ without intersecting it. More formally, given
two \newhl{watertight (connected, closed, self-intersecting-free and oriented) meshes}
%
$\Cout$ and $\Fout$ (later we generalize our method to build cages for polygon
soups, point clouds, etc.), we say that $\Cout$ \emph{nests} $\Fout$ if: 
%
(i) $\Fout$ is contained in the interior of $\Cout$, and 
%
(ii) there is a homotopy from $\Cout$ to $\Fout$ that remains at all times inside
$\Cout$'s volume. \newhl{XXX TODO: FIX XXX}
This definition extends naturally to polyhedra with voids or multiple connected components that are bounded by watertight meshes.

\begin{wrapfigure}[15]{r}{1.02in}
%
\centering
%
\includegraphics[trim=4mm 0 0mm 0mm,width=\linewidth]{figs/boxer-cages-not-cages}
%
\end{wrapfigure}
%
The inset figure shows examples of cages that do (top) and do not (bottom) nest
a 2D example (black outline). Given a $\Cinp$ that does not nest $\Fout$, we
look for the best perturbation $\Cout$ of $\Cinp$ that does nest $\Fout$; where
``best'' is measured by an application-dependent fitness energy $E(\Fout,
\Cinp, \Cout)$, penalizing, for example, total volume for tight fitting cages
for efficient multiresolution solvers; distortion for simulation or collision
handling; or violations of symmetry and quad-planarity for interactive
deformation cages.  Finding $\Cout$,  entails solving the \emph{nested cage
problem}:

Given an embedded polyhedron $\Fout$, a polyhedron $\Cinp$ homeomorphic to $\Fout$, and a
fitness function $E$, minimize $E(\Fout, \Cinp, \Cout)$ over all embeddings
$\Cout$ of $\Cinp$ that nest $\Fout$.


\begin{figure}
  \includegraphics[width=\linewidth]{figs/torus-counterexample}
  \caption{Some coarse-fine mesh combinations are impossible. The
  \emph{Cs\'asz\'ar torus} is too coarse to nested around the
  \emph{Squiggly torus}'s complex handle. While embedding the \emph{Squiggly
  torus} \emph{inside} the \emph{Cs\'asz\'ar torus} is possible (right), this
  is not a validly \emph{nesting} cage.}
  \label{fig:torus-counterexample}
\end{figure}

This problem in its full generality is intractable. For
single polygons in 2D, and simply-connected meshes in 3D, at least one nested
embedding $\Cout$ always exists, but once $\Fout$ is allowed to have nontrivial
topology, counterexamples exist where it is impossible to embed $\Cinp$ in a way
that nests $\Fout$ (see \reffig{torus-counterexample}). The decision
problem of whether any nesting of $\Cinp$ around $\Fout$ exists can be 
shown to be NP-complete~\cite{CagesNP}.
Fortunately, in practice $\Cinp$ and $\Fout$ are not completely arbitrary; any
reasonable decimation scheme will keep $\Cinp$ morphologically similar, and
roughly aligned, to $\Fout$, and these favorable initial conditions can be
leveraged by heuristics.

\paragraph{Contribution.} We present a practical algorithm for solving the
nested cage problem on meshes typically used in applications such as physical
simulation and geometric modeling. Our method 
remains completely \emph{agnostic} to the decimation scheme used to create
$\Cinp$ and fitness function $E$. Briefly, the method first flows $\Fout$ inside $\Cinp$
along $\Cinp$'s signed distance field; once inside, we rewind the flow while
deforming $\Cinp$ to minimize $E$; contact forces guarantee that the coarse cage
always encloses $\Fout$ at each reverse-flow time step. This process can be
generalized to building an entire multiresolution hierarchy around $\Fout$.
Although our method is not guaranteed to always find a solution $\Cout$,
especially for very coarse or nearly-self-intersecting $\Fout$ (see
\refsec{conclusion}
for discussion of failure cases), we tested our algorithm on an extensive zoo
of example meshes (see \refsec{results}).


\section{Related methods and special cases}

\newhl{Our method solves the nested cage problem for the large
variety of input meshes shown in \refsec{results} \emph{while} interoperating
with arbitrary problem-specific decimation algorithms and fitness functions; however}
many related approaches have been proposed that work for special cases, or
relax requirements imposed above.

If the algorithm is free to generate the coarse cage $\Cinp$, rather than
accepting it as input from the user, the problem becomes substantially easier,
and has been extensively studied:
%
\begin{figure}[b]
\includegraphics[width=\linewidth]{figs/warrior-poisson-vs-cgal}
\caption{Contouring requires aggressive spacing between distance-field
isolevels to produce valid nesting. Semantically distant parts \emph{fuse}
together, destroying shape-awareness, visualized by a pseudocoloring of a
Poisson solution computed on each level.} \label{fig:warrior-poisson}
\end{figure}

\paragraph{Bounding polytopes.}
%
Most naively, one can simply take $\Cinp$ to be a canonical bounding polytope
(e.g.\ box, KDOP, convex hull). An extension would be to ``shrink-wrap'' the
bounding polytope around $\Finp$ to minimize $E$. This idea has been explored
before \cite{Peterhans:2012,Wang:2013:HPE}, but no full solution has been
developed.
%
This is presumably because---while this approach works well for convex
$\Finp$---it will not give a good fit for meshes with nooks \newhl{or} concavities. It is
also not clear how to find a bounding polytope for meshes of nontrivial
topology.
%
Another approach might be to stitch overlapping convex volumes
\cite{Xian:2012tv} using mesh boolean operations, but cage topology and
\emph{quality} become difficult to control without sacrificing control of
resolution.

\paragraph{Offset surfaces for simulation meshes.}
%
Representing $\Finp$ as an implicit surface and \emph{contouring level
sets} is a popular method of creating offset surfaces that nest $\Finp$,
particularly in the context of simulations that use a ``simulation mesh'' as a
proxy for very fine rendered geometry~\cite{Campen:2010}. The challenge is not
just defining a scalar function with an isosurface enveloping the object, but
also contouring it with a piecewise-linear triangle mesh without (i) cutting
corners too harshly and intersecting $\Finp$ or (ii) resorting to fine
resolution, nullifying any performance gains of using the cage.
%
Xu et al~\shortcite{Xu:2014:SDF} define a robust signed distance field, but
pass the field to an off-the-shelf contour mesher without any attempt to
guarantee non-intersection with the input.  Similarly, Shen at
al~\shortcite{Shen:2004:IAI} iteratively refine a moving least squares
iso-surface to enclose an existing model tightly via a global scaling
parameter.
%
Some applications, such as collision detection, can make use of implicit
representations without intermediary meshes,
%
but when a mesh is needed it is not enough that the \emph{continuous}
isosurface does not intersect the input model since generic contouring will
invalidate this.
%
A large enough iso-value or
small-enough resolution tolerance must always be chosen to ensure
non-intersection after contouring. For large iso-values, topological control is
lost and close features are quickly merged (see \reffig{warrior-poisson}),
violating the nestedness of the cage. A similar
approach~\cite{Ben-Chen:2009:SDT} has been used for building deformation cages
around input shapes, where an offset surface is created via Poisson surface
reconstruction~\cite{PoissonSurfaceReconstruction06} (we compare against this
approach in ~\reffig{dane-vs-ben-chen}). \newhl{While the tightness of these cages could be controlled with post-hoc shrinking of the cage, the method can introduce topology changes that are not easily remedied.}
%
For meshes with well defined feature curves, it may be possible to merge coarse
on-surface triangulations of patches \cite{Xian:2013}, though strict nesting is
not sought or guaranteed.

\begin{figure}
  \includegraphics[width=\linewidth]{figs/dane-horse-vs-ben-chen-vs-sander}
  \caption{Methods like \protect\cite{Ben-Chen:2009:SDT} use coarse cages as
  computational workhorses to reduce complexity. Their iterative offset
  heuristic \newhl{does not allow control of the cage topology.}
% oversmoothed, loosely fitting cages.
  %
  Meanwhile, \protect\cite{Sander:2000:SC} attempt to
  maintain nestedness during greedy decimation with local constraints. Cages
  \newhl{are loosely-fitting}, but more importantly self-intersections (orange) and
  intersections with the input model (yellow) eventually accumulate.}
  \label{fig:dane-vs-ben-chen}
  \vspace*{-5mm}
\end{figure}

\textbf{Progressive decimation} of $\Finp$ using edge collapses, taking care
to place new vertices on the exterior of the current volume by solving a system
of inequality constraints~\cite{Sander:2000:SC}, has found some success in
real-time rendering and collision detection \cite{Platis:CGF}.  Although
the \emph{vertices} of these ``progressive hulls'' are guaranteed to lie
outside of $\Finp$, edges and faces of $\Cinp$ might still
intersect $\Finp$ (and $\Cinp$ might globally self-intersect); see
\reffig{dane-vs-ben-chen}. We tested this method on the entire ``zoo'' of
examples shown in \reffig{zoo} and the supplementary material, and of the 26
examples there, in only one case was the entire hierarchy free of such
intersections. We also observe that for coarse cages these hulls tend to be
more loose-fitting than cages produced by methods that optimize the cage shape
globally (see \reffig{dane-vs-ben-chen}).  Self-intersections resulting from
edge collapses can be corrected with post-hoc mesh repair~\cite{Deng:2011vr},
but this technique relies on temperamental 3D tetrahedral meshing in tight
regions near overlaps and does not consider face\newhl{- or }edge-intersections with
the input model.

\textbf{Voxelization} of $\Finp$ will create a nested cage, provided that
the resolution is chosen fine enough to avoid topological
artifacts (as in the case of contouring implicit functions). Naive voxelization
yields dense, inefficient cages \cite{Mehra:2009:AMS}, possibly improved by
progressive decimation~\cite{Xian:2009} or mesh booleans~\cite{Xian:2015}
though also inheriting their respective drawbacks.

\paragraph{Mesh untangling.}
%
All of the above methods require relinquishing some degree of control over
decimation to the nested cage algorithm; algorithms that accept an arbitrary
pre-decimated cage $\Cinp$ are less well-developed. \emph{History-free} cloth
collision response methods (e.g.\
\cite{Baraff:2003:UC,Volino:2006:RSC,Wicke:2006,Ye:2012:ICM}) could be used to
separate $\Cinp$ from $\Finp$ if intersections are not too severe; otherwise
such methods quickly get stuck in local minima.

\paragraph{\newhl{Outward Flow.}}
Before settling on an inward flow of $\Finp$, we experimented with an outward
flow of $\Cinp$ away from $\Finp$, along its signed distance field.
%
For convex meshes this works well\newhl{. However, extensive testing revealed
that this approach suffers from several difficulties on more complex geometries: 1)
Flowing vertices along the signed distance field of $\Finp$ is not guaranteed
to resolve all face-face collisions between the fine and coarse meshes; the
flow is most robust when the flowing elements are small and the signed distance
field does not have too many fine features. Flowing the coarse mesh outward is
more fragile than flowing the fine mesh in as both of these factors are less
favorable; 2) when flowing the fine mesh inside the coarse mesh we do not need
to handle self-collisions within the fine mesh, whereas if inflating the coarse
mesh, we do. Adding collision constraints aggravates the fragility of the
outward flow; the constraints can prevent a flowing face from ever leaving the
volume of the fine mesh (see Figure~\ref{fig:knight-exp-vs-sh}).}

\begin{figure}
  \includegraphics[width=\linewidth]{figs/knight-exp-vs-sh}
  \caption{\newhl{Shrinking the fine mesh inside the coarse quickly removes overlaps
  (yellow). Expanding the coarse mesh requires self-contact handling (e.g.\
  between arms and body), blocking progress.}}
  \label{fig:knight-exp-vs-sh}
  \vspace*{-5mm}
\end{figure}

\paragraph{Multigrid.}
%
Methods based on regular grids or lattices naturally support multiresolution
numerical methods.
%
McAdams et al.\ \shortcite{McAdams:2011}
successfully employ multigrid on voxelized shapes to animate
volumetric elastic characters.
%
The main downside of grid-based methods is their traditionally poor handling of
irregular boundaries.
%
Specialized multigrid methods using adaptive octree\newhl{s} to handle complex
boundaries for fractures \cite{Dick:2011} and fluids \cite{Ferstl:2014} exist,
but are not immune to troubles of regular grids: boundaries must be 
represented at fine grid levels to avoid aliasing and retain the input's
topology.
%
To avoid \emph{fusing} together geodesically distant parts, the
shape must be modeled in an accommodating pose. If remodeling is not
possible, the grid size must be chosen smaller than not just the smallest
features but also the \emph{smallest void} between features.
%
One solution is to replicate cells in close areas and manage
adjacencies when two or more replicated patches merge and split
\cite{Teran:2005:CSS,Nesme:2009:PTE,Sykora09}.
%
Cell replication is not only difficult to realize robustly, but also riddles
multigrid numerical methods with expensive, SIMD-breaking boundary handling
code\newhl{,} detracting from the performance gains of memory-efficient regular grids
\cite{Demmel04}.

Multigrid on unstructured grids or tetrahedral meshes is more
temperamental, and constructing each level requires care
\cite{fish1995efficient}.
%
Geometric multigrid schemes typically coarsen an input tetrahedral mesh by
removing vertices, attempting to connect those remaining in a reasonable way
\cite{guillard1993,Adams:1999:PMS}. Special care is required to maintain any
semblance of the original boundary \cite{Brune:2011}, essentially devolving
into constrained Delaunay tessellation with no guarantee that the coarsening
will not \emph{eat away} large portions of the domain.
%
In many scenarios, the boundary of the domain is assumed to be only as
irregular as the coarsest layer, simplifying level design \cite{feng1997non}.

%\cite{fish1995efficient} just an example of a (nonlinear) multigrid solver
%which expects the user to provide (nested?) tetrahedral meshes as input.
%
%\cite{feng1997non} more complicated restriction operator for Galerkin multigrid
%on non-nested meshes, though it seems it's expected that the surfaces of each
%mesh coincide: user provides meshes. I'm not sure how much better this
%restriction would really be compared to simple barycentric one.

A second group of methods generates a multiresolution hierarchy using
decimation. Unmodified mesh decimations \cite{Garland:1997:SSU} have been used
for adaptive simulations for visco-elastic solids \cite{Debunne:2001:DRD}.
%
Such non-nested cages require extrapolation or one-to-many mappings to
\emph{prolongate} solutions on the coarse levels to finer levels.
%
We show cases where this fails to converge for common linear systems on
irregular domains, and we show better convergence for strictly nesting cages
where prolongation is a purely linear interpolation.

Algebraic alternatives also exist
\cite{ruge1987algebraic},
but are recommended only when geometric information is not available
\cite{falgout06}.
%
Recently, \cite{Krishnan:2013:EPL} proposed a
multigrid preconditioner for Laplace-based systems on images and
surfaces.
%
This method combines the elegance of algebraic techniques with some geometric
information derived from the characteristics of the Laplacian matrix, but is
limited to special problems.


%Alternatives to geometric multigrid: algebraic multi-grid (cite Brandt) or
%algebraic techniques for special problem \cite{Krishnan:2013:EPL} (doesn't
%generalize beyond ``M-matrices'')

%\cite{CGCDP:2002} assume a subdivision lattice containing the deforming shape
%as part of the input. Presumably this cage is modeled manually.

\paragraph{Simplification.}
%
The majority of surface mesh decimation techniques aim to preserve outward
appearance with lower and lower mesh resolution
\cite{Hoppe:1996:PM,Garland:1997:SSU,Melax98}.
%
Along these lines, \cite{gumhold2003intersection} output
decimations free of \emph{self-}intersections, ensuring for example that a
character's clothing stays outside its body.
%
In contrast, our method assumes self-intersection free input decimations and
transforms them into nested layers, with no intersections \emph{across}
hierarchy layers.
%
Recent work has considered more elaborate decimation goals than appearance such
as preserving haptic sensations \cite{Otaduy:2003:SPS}.
%
\cite{Otaduy:2003:CDH} also demonstrate how to combine mesh decimations and bounding
volume hierarchies to achieve faster collision detectio\newhl{n w}ithout
affecting visual appearance.
%
Our nested cages are hierarchical decimations strictly containing the input
shape, ensuring strictly conservative collision detection.
%
Rather than work against the sophistication of existing shape decimation
techniques, we complement them. Our method \newhl{takes} arbitrary decimations as input and
nests them as a post process.

\paragraph{Interference-aware processing.} We credit
\cite{ContactAwareModeling:2011} for their ground-breaking introduction of
contact handling to mainstream geometry processing. Their work inspires us to
consider the contact and collisions tool-set familiar to physically based
simulation in our geometry processing task.
%
In this task, our novel flow is essential for finding a feasible starting
state.

\section{Method}
\label{sec:method}

% establish input and output
The input to our method is a sequence of $k+1$ potentially overlapping
triangles meshes $(\V_0,\F_0),(\V_1,\F_1),\dots,(\V_k,\F_k)$, where $\V_i$ is a
list of 3D vertex positions and $\F_i$ is a list of triangle index triplets.
%
In general, we only assume that each mesh \emph{approximates} the surface of the same
solid object.
%
In a typical scenario, $(\V_0,\F_0)$ is a high-resolution \emph{original} mesh and
$(\V_1,\F_1),\dots,(\V_k,\F_k)$ are decimations of decreasing resolution.
%
We require that each input mesh is \emph{watertight} \cite{Dey:2003jf}:
%
free of open boundaries, non-manifold elements, or
self-intersections.\footnote{Our input assumptions are stricter than the
\emph{solidness} of \cite{Bernstein:2013:PHH}. That definition permits
non-manifold ``shared'' vertices and edges which confuse decimators.}
%
Depending on the application, troublesome input meshes may be \emph{cleaned} as
a pre-process using available tools (e.g.\
\cite{Attene:2010vv,Jacobson:WN:2013}).

The output of our method is a new sequence of $k$ lists of vertex positions 
$\V'_1,\dots,\V'_k$ such that all points on $(\V'_{i-1},\F_{i-1})$ are
strictly
\emph{inside} $(\V'_i,\F_i)$ for $i=1\dots k$, letting $\V'_0 = \V_0$ (see
\reffig{2d-nested-layers-notaion}).

We opt to alter only the vertex positions (\emph{geometric embedding}) of each
mesh, and not the number or connectivity of vertices ($\F_i$ are unchanged). Among
other benefits, this design decision ensures that the number of vertices in each
mesh is exactly maintained.

The \emph{nesting} property of the output meshes is easily \emph{verified} by testing
that the winding number of at least one vertex of $\V'_{i-1}$ with respect to
$(\V'_i,\F_i)$ is positive and no intersections exist between
$(\V'_{i-1},\F_{i-1})$ and $(\V'_i,\F_i)$. 

Now we describe a general method that guarantees this nesting property while
optimizing any problem specific energy (e.g.\ distance between meshes, mesh
volumes).

Our method operates recursively on two meshes of the sequence at a time: we
compute $\V'_i$ by considering only the solution to the previous level
$(\V'_{i-1},\F_{i-1})$ and initial mesh $(\V_i,\F_i)$. 
%
Without loss of generality, let us simplify our notation and consider computing
$\V'_1$ from $(\V'_0,\F_0)$ and $(\V_1,\F_1)$. We refer to the $0$th
mesh as the ``fine mesh'' and the $1$st mesh as the ``coarse mesh''.
%
Computing new coarse mesh vertex positions $\V'_1$ involves three phases:
\emph{flow} the fine mesh until fully inside the coarse mesh, \emph{re-inflate}
the fine mesh to its original embedding while \emph{pushing} the coarse mesh
out of the way, and finally \emph{optimize} the coarse mesh embedding (see
\reffig{2d-pipeline}).

\subsection{Flow}
\label{sec:flow}

Without loss of generality, the input fine mesh $(\V'_0,\F_0)$ and coarse mesh
$(\V_1,\F_1)$ are free of \emph{self}-intersections,
%
but, in general, the fine mesh will overlap the coarse mesh: some triangles of
the fine mesh will intersect those of the coarse mesh, implying that some
portion of $(\V'_0,\F_0)$ lies outside of $(\V_1,\F_1)$. Equivalently, there
exists a non-empty set of points on $(\V'_0,\F_0)$ with \emph{positive} signed
distance with respect to $(\V_1,\F_1)$ \footnote{Assuming a negative
inside, positive outside convention.}.
%
Our idea is to find a new embedding $\widehat{\V}_0$ that \emph{minimize
signed distance} integrated over all points $\p$ of the fine mesh to the
closest point $\q$ on the coarse mesh:
\begin{align}
\label{equ:sd-energy}
 Φ(\widehat{\V}_0) &= ∫_{(\widehat{\V}_0,\F_0)} s(\p) u(\p) \,dA,\\
& u(\p) = \left\| \p - \argmin_{\q \in (\V_1,\F_1)} \|\p-\q\|\right\|,\\
& s(\p) = \begin{cases}
  1 & \text{ if $\p$ is outside $(\V_1,\F_1)$},\\
  0 & \text{ if $\p$ is exactly on $(\V_1,\F_1)$},\\
  -1 & \text{ if $\p$ is inside $(\V_1,\F_1)$},
\end{cases}
\end{align}
where $u(\p)$ is the \emph{unsigned} distance from $\p$ to the coarse mesh and
$s(\p)$ modulates by the appropriate sign.

\alec{Are we actually using signed \emph{squared} distance?}

It is important that $\p$ consider \emph{all points} on the fine mesh, not just
vertices. Though \emph{all vertices} in $\widehat{\V}_0$ may lie inside
the coarse mesh, parts of edges and facets might still lie outside (see
\reffig{2d-cutting-corner}). It is also important that $\q$ consider \emph{any
point} on the coarse mesh, as the nearest vertex may be arbitrarily farther
away than the closest point along an edge or facet.

We may immediately write our energy as sum of integrals over each triangle 
$\{a,b,c\}$ in $(\widehat{\V}_0,\F_0)$:
\begin{equation}
Φ(\widehat{\V}_0) = ∑_{\forall \{a,b,c\}} ∫_{\p \in \{a,b,c\}} s(\p) u(\p)\,dA.
\end{equation}

We minimize the energy by taking small steps opposite the gradient direction
for each vertex position $\widehat{\vv}$ in $\widehat{\V}_0(t)$ as a function
of a hypothetical \emph{time} variable $t$:
\begin{equation}
\dd{\widehat{\vv}}{t} = -\Grad_{\widehat{\vv}} Φ(\widehat{\V}_0).
\end{equation}
%
By following this gradient, we \emph{flow} the fine mesh vertices, until
$\widehat{\V}_0(t)$ is fully inside the coarse mesh.

Our continuous energy in \refequ{sd-energy} is similar to the data terms found
in non-rigid registration techniques \alec{cite non-rigid ICP paper}. However,
the sign modulator $s(\p)$ makes an important difference. Minimizing unsigned
(positive) distances would encourage points toward the surface of the coarse
mesh. Instead, by encouraging negative distances, points are attracted to the
medial axis \emph{within} the coarse mesh.

To differentiate the 
continuous energy in \refequ{sd-energy}, we
%
first approximate the continuous integral using a discrete set of quadrature 
evaluation points $\p_i$ and corresponding weights $w_i$ for each triangle:
\begin{align}
Φ(\widehat{\V}_0) &\approx ∑_{\forall \{a,b,c\}} ∑_i w_i s(\p_i) u(\p_i),\\
& \p_i = 
\lambda_a \vv_a + 
\lambda_b \vv_b + 
\lambda_c \vv_c,
\end{align}
where $\lambda_a,\lambda_b,\lambda_c$ are the barycentric coordinates locating $\p_i$
in the triangle $\{a,b,c\}$.
%
We use second-order quadrature rules and see diminishing returns with more
exact schemes.
%
However, the difficulty of differentiating $u(\p_i)$ remains. To tackle this,
we adapt the successful iterative closest point approach of non-rigid
registration techniques. Namely, we assume that the closest point $\q_i^*$ to each
$\p_i$ remains constant during each small time step. Likewise, we assume that
the sign $s(\p_i)=s_i^*$ remains constant during each time step.

Now, the we can derive a
gradient for the $j$th mesh vertex $\widehat{vv}_i$. 
Without loss of generality, if we assume triangle indices ${a,b,c}$ are always
\emph{rotated} so that $a=j$, if any, then 
\begin{align}
\dd{\widehat{\vv}_i}{t} &\approx -\Grad_{\widehat{\vv}_i} 
∑_{\forall \{a,b,c\} | a = j} ∑_i w_i s(\p_i) u(\p_i),\\
&= -\Grad_{\widehat{\vv}_i} 
∑_{\forall \{a,b,c\} | a = j} ∑_i w_i s_i^* \|\p_i -\q_i^*\|,\\
&= -
∑_{\forall \{a,b,c\} | a = j} ∑_i w_i s_i^* \Grad_{\widehat{\vv}_i} \|\p_i -\q_i^*\|,\\
&=-
∑_{\forall \{a,b,c\} | a = j} ∑_i w_i s_i^* \Grad_{\widehat{\vv}_i}
\|
\lambda_a \vv_i + 
\lambda_b \vv_b + 
\lambda_c \vv_c
-\q_i^*\|,
\end{align}
\alec{whhhhhhaaat. My gradient has run amok. How come I'm not getting that we
have a weighted sum of directions toward the closest points to each quadrature
point?}
\begin{align}
&=-
∑_{\forall \{a,b,c\} | a = j} ∑_i w_i s_i^* \g_i,\\
&\text{ where } \g_i = \begin{cases}
\frac{\p_i -\q_i^*}{\|\p_i -\q_i^*\|} & \text{ if } \|\p_i
-\q_i^*\|<\epsilon,\\
\n(\q_i^*) & \text{ otherwise },
\end{cases}
\end{align}
where $\n(\q_i^*)$ is the unit normal at $\q_i^*$. This normal is chosen with
care. To ensure that it points \emph{inside} the coarse mesh, we determine if
$\q_i^*$ lies near a vertex, near an edge or in the middle of a triangle and
use an interior angle weighted average of incident triangle normals, a uniform
average of incident triangle normals or triangle normal,
respectively \cite{Baerentzen:2005:SDC}.

We iterate this flow stepping with magnitude proportional to a small ``time
step'': $∆t \approx XXXXXX$ \alec{what do we use?}. After each time step we
recompute signs $s_i^*$ and closest points $\q_i^*$ all quadrature points. We
terminate if all signs are negative \emph{and} no intersecting faces are found
between $(\widehat{\V}_0,\F_0)$ and $(\V_1,\F_1)$.

There is no formal, provable guarantee that this flow will succeed. Indeed, in
rare, difficult cases the flow converges without moving the fine mesh fully
inside the coarse mesh. This may occur if the medial axis of the coarse mesh
lies outside of the fine mesh. This is especially rare if the input sequence
results from decimation as decimation effectively \emph{prunes} and simplifies
a mesh's medial axis.

Rather than admit defeat, we propose a additional step which alleviates many
occurrences of this problem. We reverse the picture and expand the coarse mesh,
flowing it away from the current fine mesh along its signed distance field.

\alec{Might want to answer question: why not just expand the coarse mesh or
always do both?}

\alec{Relationship to David's penetration depth energy: needs intersection free
state; and Untangling Cloths' volume minimization: might push coarse mesh
\emph{inside} fine mesh instead.}

\alec{Should state that we 1) don't care if $(\widehat{\V}_0,\F_0)$ becomes
self-intersection or 2) cannot become self-intersecting.}

\subsection{Re-inflation}
\label{sec:reinflation}

At this point, we have flowed the fine mesh vertices so that
$(\widehat{\V}_0,\F_0)$ is fully inside $(\V_1,\F_1)$. We now restore the fine
mesh to its original vertex positions $\V_0$, detecting and resolving collisions with
the coarse mesh along the way.

To do this, we iterate through the vertex positions computed for each time step
during the flow described in \refsec{flow} \emph{in reverse}.
%
Doing so without moving the coarse mesh vertices would immediately result in
fine-coarse intersections.

We can describe each reverse step in our flow in terms of a displacement per
time step, that is, in terms of \emph{velocities}. For the fine mesh, the
positions after the next reverse time step are known, and thus so are its
velocities:
\begin{align}
\widehat{\V}_0(t-∆t) &= 
\widehat{\V}_0(t) - ∆t \widehat{\U}_0(t),\\
\widehat{\U}_0(t) &= 
\frac{\widehat{\V}_0(t) - \widehat{\V}_0(t-∆t)}{∆t}
\end{align}
where $\widehat{\V}_0(t)$ are the vertex positions of the fine mesh at ``flow
time'' $t$ with the positions returning to their input positions at time zero
$\widehat{\V}_0(0) = \V'(0)$, and $\widehat{\U}_0(t)$ are the instantaneous
per-vertex 3D velocity vectors at time $t$.
%
\alec{The signs are funny since we're flowing backwards.}

For the coarse mesh, the positions $\bar{\V}_1(t-∆t)$---and in turn velocities
$\U_1(t)$---are not fixed. In general, there are an infinite number of
\emph{feasible} choices of $\U_1(t)$ so that the repositioned coarse mesh
$(\bar{\V}_1(t-∆t),\F_1)$ remains free of intersections with itself and with the
\emph{re-inflating} fine mesh $(\widehat{V}_0(t-∆t),\F_0)$.
%
To regularize this problem, we minimize the change in position, or equivalently
minimize the magnitude of velocities.
%
Our reverse time step problem becomes:
%
\begin{align}
\label{equ:simulation}
&\argmin_{\U_1(t)} \left\|\U_1(t)\right\|^2,\\
&\text{ subject to: }\\
&(\bar{\V}_1(t-∆t),\F_1) \text{ does not intersect itself},\\
&(\bar{\V}_1(t-∆t),\F_1) \text{ does not intersect } (\V_0(t-∆t),\F_0).
\end{align}

\alec{This is just disallowing instantaneous collisions. We actually model a
harder problem: no continuous collisions. ``Why?''}

By reformulating our problem in a manner familiar to physically based
simulation, we may leverage state of the art contact detection and response
methods.
%
Abstractly, we can treat these methods as a ``black box'', providing it
the fine mesh $(\V_0(t),\F_0)$ and coarse mesh $(\bar{\V}_1(t),\F_1)$ at the end of the
reverse flow time step and the desired velocities $\U_0(t)$ and $\U_1(t)$. 


There remains one interesting twist. Our problem requires the fine mesh to
return exactly to its original positions. In physically-based simulation
parlance, this is tantamount to assigning the fine mesh \emph{infinite mass}.
%
Many methods are fundamentally incapable of handling such constraints
\alec{cite synchronous penalty-based methods?}.
%
Instead, we experimented with two methods.

First, we adapt the ``velocity adjuster'', \alec{there's a better word than
``adjuster''} surface tracking method of \cite{Brochu:2009} to deal with
infinite masses. \alec{What did we have to change in El Topo? We decrease the
time step and retry assuming linear motion?}

In difficult cases, too many time step subdivisions are needed, suggesting
failure to make progress in finite time. To handle these hard cases, we fall
back on a second, more robust but slower method: speculative asynchronous
contact mechanics \cite{Ainsley:2012:SPA}. This method is an extension of the
only known method to guarantee intersection prevention and positive progress
for \emph{finite mass} objects \cite{Harmon:2009}. Our infinite masses pose an
interesting stress test for this method, but we see success, albeit at a slower
pace.

\subsection{Optimization}

So far we have allowed the coarse mesh positions to \emph{drift} as they are
\emph{pushed} outward by the re-inflating fine mesh. Afterwards, the embedding
of the coarse mesh $(\bar{\V}_1(0),\F_1)$ is strictly outside the fine mesh,
but may have strayed from an \emph{optimal} embedding.

Our final phase improves the positioning of the coarse mesh. The formulation is
general and allows the user to define the optimality metric for the
domain-specific problem at hand. 

Given a differentiable energy $E(\V'_1)$ and a current, feasible embedding
$(\V'_1,\F_1)$ we set up the following ``simulation'', reminiscent of
\refequ{simulation}:
\begin{align}
&\argmin_{\U'_1} \left\|\U'_1 - δ \Grad E(\V'_1) \right\|^2,\\
&\text{ subject to: }\\
&(\V'_1+\U'_1,\F_1) \text{ does not intersect itself},\\
&(\V'_1+\U'_1,\F_1) \text{ does not intersect } (\V'_0,\F_0).
\end{align}
where $δ \approx XXXXXX$\alec{what do we use?} is a small step size
parameter and $\Grad E(\V'_1)$ is the gradient of the user supplied energy with
respect to vertex positions. We solve this problem iteratively, setting $\V'_1
\leftarrow \V'_1 + \U'_1$ after each step until convergence using the same
black box solvers as in \refsec{reinflation}.

We experimented with a few different choices of optimality measures.
One natural choice is to minimize the coarse layer's volume, preferring a
tighter fit. In this case, we define:
\begin{align}
E_\text{volume}(\V'_1) &= ∫_{(\V'_1,\F_1)} 1 \,dA,\\
\Grad E_\text{volume}(\V'_1) &= \N_1,
\end{align}
where $\N_1$ are the per-vertex, area-weighted normals \alec{Should just be a
citation for this}.

\alec{what other energies? Dirichlet? ARAP? symmetry? planarization? Do we
really need to include definitions and gradients for all of these?}

\alec{Need to address: ``why not optimize while reinflating? why optimize only
at the end?'' We prioritize feasibility and performance before
quality...hmmm...that doesn't sound good.}

\alec{Need to address: ``why not try avoid collisions during decimation?''}

\alec{Might want to address: ``why accept pre-decimated meshes? Why not include
decimation as part of pipeline''?}

\alec{Need to address: ``why not shrink wrap a mesh from well outside?''}

\alec{We could handle energies which measure across layers. Minimize squared
separation distance between adjacent layers: might pay to expand middle layer
to balance space between fine and coarse layers. We can model this as one large
optimization problem, and recommend that one still proceed with a block
coordinate descent freezing all but one layer iteratively.}

\section{Results and applications}
\label{sec:results}

We implemented a prototype of our method as a serial \textsc{Matlab} program.
%
We report timings of our unoptimized code for a few representative examples in
\reftab{timings} recorded on an iMac Intel Core i7 3.5GHz computer with 8GB
memory.
%
As expected the bottleneck is the collision-free re-inflation step.
%
We experimented with a wide variety of shapes, ranging from CAD models,
characters, and scanned objects (see \reffig{zoo}). By default we compute
layers so that each coarser layer has $2^{-\sfrac{2}{3}}$ times as many facets as
the previous finer layer, a ratio chosen so that resulting tetrahedral
meshes will have approximately $8$ times fewer elements. For most
meshes we compute seven layers, with fewer for lower resolution inputs.
%
In our supplemental material, we attach all input models, corresponding output
cages, and a small program to visualize volumetric slices.

% !TEX root = winding.tex
\begin{table*}
\centering
\ra{1.2}
\setlength{\tabcolsep}{5.5pt}
\rowcolors{2}{white}{lightbluishgrey}
\begin{tabular}{l r r r r r r r r r r r r r r r r r}
\rowcolor{white}
Model name  & Input \#F & \#F for layers & Avg. shrink time per layer & Avg.
re-inflation time & Energy type\\
\midrule
Octopus & 100,000 & 3,936 \ 6,248 \ 9,920 \ 15,748 \ 25,000 \ 39,684 \ 62,994 & 11s & 227s & Volume \\
Pelvis & 40,316 & 1,586 \ 2,516 \ 3,998 \ 6,346 \ 10,078 \ 15,994 \ 25,394 & 11s & 460s & Volume  \\
Model9 (Warrior?) & 26,658 & 1,048 \ 1,666 \ 2,644 \ 4,198 \ 6,664 \ 10,578 \ 16,790 & 37s & 497s & Volume  \\
Anchor & 10,778 & 678 \ 1,068 \ 1,696 \ 2,694 \ 4,276 \ 6,788 & 8s & 34s & Volume \\
Horse & 39,696 & 1,562 \ 2,480 \ 3,936 \ 6,250 \ 9,922 \ 15,752 \ 25,006 & 12s & 321s & Surface ARAP \\
Armadillo & 12,000 & 470 \ 748 \ 1,190 \ 1,888 \ 2,998 \ 4,760 & 27s & 998s & Voumetric ARAP \\
Bunny & 34,832 & 1,371 \ 2,177 \ 3,455 \ 5,485 \ 8,708 \ 13,823 & 3s & 95s & Volumetric ARAP \\
Gargo & 13,500 & 531 \ 843 \ 1,339 \ 2,126 \ 3,375 \ 5,357 \ 8,504 & 2s & 90s & Volumetric ARAP
% new entries here:
\bottomrule
\end{tabular}
\caption{}
\label{tab:timings}
\end{table*}


Our method is agnostic to the decimator used to create the input meshes
$\Minp_1,\dots,\Minp_k$. In this way we inherit the feature set of the
decimator. 
%
\reffig{decimations}
compares using the regular mesh inducing decimator in \cite{cgal} (default for
all remaining examples) and the feature-adaptive decimator of \cite{openmesh}.
% Meshfix is mentioned at beginning of method.tex

\begin{figure}
  \includegraphics[width=\linewidth]{figs/coupling-down-decimators}
  \caption{By post-processing meshes from any existing decimator, our output
  inherits desired regularity or adaptivity of the decimation.}
  \label{fig:decimations}
\end{figure}

%\alec{INSERT ENERGY COMPARISON HERE}
%
%
%\begin{figure}
%  \includegraphics[width=\linewidth]{figs/energies_volume_surfarap.png}
%  \caption{The cage obtained by minimizing the volume energy (left)
%  does present a tight fit but may lead to mesh quality artefacts.
%  On the other hand, minimizing surface ARAP (right) preserves the mesh
%  quality of the initial (non-regular) decimation but leads to extra volume.}
%  \label{fig:energies}
%\end{figure}

To test robustness, we compare computing seven nested cages on the \emph{Bunny}
and the same model corrupted with noise in the normal direction (see
\reffig{noisy-bunny}). The resulting layers tightly hug both shapes, producing
similar outermost cages.

\begin{figure}
  \includegraphics[width=\linewidth]{figs/noisy-bunny}
  \caption{Noise added in the normal direction to the input bunny does not
  affect our ability to generate seven quality outer layers.}
  \label{fig:noisy-bunny}
\end{figure}

We also conducted stress tests to evaluate how well our method scales with the
number of layers. We nest 25 tightly fitting layers around the \emph{Horse} in
\reffig{horse-25-layers} and 50 around \emph{Max Planck} in \reffig{zoo}, top
left.
%
We purposefully continue nesting cages around the the \emph{Gargoyle} in
\reffig{zoo}, bottom left, until only eight vertices of an extremely coarse
cage remain.

\begin{figure}
  \includegraphics[width=\linewidth]{figs/octopus-poisson}
  \caption{A single multiresolution v-cycle takes 1.4 secs on this 7M-vertex
  volumetric Poisson equation in the \emph{Octopus}. With 14 more iterations
  (21 secs) the residual error matches a direct solver's (11 mins, back
  substitution only).}
  \label{fig:octopus-poisson}
\end{figure}
\begin{figure}
  \includegraphics[width=\linewidth]{figs/pelvis-diffusion}
  \caption{We solve a diffusion equation $(λ∆+I) x = b$ for various diffusion
  rates $λ$ with our nested cages in a volumetric multiresolution solver (top:
  surface values via Neumann boundary conditions, bottom: slice through
  tet-mesh volumes).}
  \label{fig:pelvis-diffusion}
\end{figure}

Constructing our nested cages can be considered expensive \emph{precomputation}
for a multiresolution linear system solver.
%
However, once cage meshes are computed and their interiors are meshed with
tetrahedra (e.g.\ using \cite{tetgen}),
the volumetric multigrid solver is sleek and memory efficient.
%
A single multiresolution V-cycle for a Poisson equation with homogeneous
Dirichlet boundary conditions inside the volume of the \emph{Octopus} with over
seven million vertices takes 1.4 seconds using 2GB max memory. With 14 more
V-cycles the solution converges for a total time of 21 seconds (see
\reffig{octopus-poisson}).
%
In contrast, \textsc{Matlab}'s backslash operator thrashes when using over 22GB
of memory finishing in over 16 minutes.
%
\textsc{Cholmod}'s Cholesky factorization with reordering is mildly better than
\textsc{Matlab}, solving via backsubstitution in 10 minutes, but suffers from
similar memory issues during factorization, which takes over
an hour using 17GB max memory, due to high fill-in.
%
In terms of precomputation, our time consuming cage computation lives at a much
earlier stage than system-matrix factorization: before determining constraints
or boundary conditions and before even choosing the particular system being
solved.
%
This is even true for inhomogeneous systems where local metrics are
varying between solves. In this case, the internal tet-meshing might need to be
recomputed (seconds for \emph{Octopus}), but our boundary cages can be reused.
%
Since all fine mesh vertices are inside coarse-mesh tetrahedra, we use linear
interpolation for prolongation and its transpose for restriction
\cite{Demmel04}.

\begin{figure}
  \includegraphics[width=\linewidth]{figs/armadillo-heat-plot}
  \caption{We solve for smooth geodesics over volumetric tetrahedral mesh
  inside the \emph{Armadillo}. Using naive overlapping and
  shrinking decimations leads to divergence unless a very large number of
  relaxation iterations is used.  Ours is always convergent.}
  \label{fig:armadillo-heat-plot}
\end{figure}
\begin{figure}
  \includegraphics[width=\linewidth]{figs/frankenstein-physics-02}
  \caption{Extracting an outer hull with Jacobson et al~\protect\shortcite{Jacobson:WN:2013} fails to coarsen the domain (top
  left). Contouring a distance field achieves nesting at a large
  iso-level but fuses the legs (top center). Our coarse cage fits the input
  tightly and provides a reduced domain for real-time physics.}
  \label{fig:frankenstein-physics}
\end{figure}

\begin{wrapfigure}{r}{1.02in}
%
\centering
%
\includegraphics[trim=0 0 0 3mm,width=\linewidth]{figs/pelvis-plot}
%
\end{wrapfigure}
%
As we do not alter the core iterative nature of multiresolution, we benefit
from its flexibility. For example, we may quickly change the diffusion rate in
a heat equation solved in the volume of the \emph{Pelvis} (see
\reffig{pelvis-diffusion}). 
%
Factorization based solvers, in general, scrap previous precomputation after
such a global change to the system matrix. 
%
Multiresolution hardly notices, and previous solutions become warm starts.
%
We employ Neumann boundary conditions and notice that naively decimating the
input mesh leads to a divergent solver (inset), agreeing with previous analysis
that nesting is important for such boundary conditions \cite{chan1999boundary}.
%
Because naive decimations do not nest eachother, the prolongation operator must
extrapolate for fine mesh vertices lying outside the coarse domain.
%
It \emph{may} be
possible to tweak extrapolation parameters to handle these cases with naive
decimation, but an automatic method for determining correct extrapolation for
convergence is not obvious.
%
For comparisons, we tried: linear extrapolation from nearest tet, constant
interpolation of nearest vertex, linear interpolation of the closest point on
nearest face. In our figures we compare to the most favorable choice.

On less challenging domains, naive decimations can be used, but may require
many relaxation (a.k.a.\ smoothing) iterations on each level of each v-cycle.
%
In \reffig{armadillo-heat-plot}, we compute smooth geodesics via two Poisson
equations in the volume of the \emph{Armadillo} \cite{Crane:2013:RFV}.
%
Using our nested cages, the multiresolution solver converges independent of the
number of relaxation iterations used (typically fewer relaxation iterations and
more v-cycles is preferable).
%
In contrast, multiresolution using naive non-nested decimations converge when
using a large number of relaxation iterations, and then at a rate equivalent to
single relaxation iteration with our meshes.

A single enclosing cage is useful for creating a lower dimensional volumetric
domain for elastic simulation of an input model that is either too high
resolution or complicated by meshing imperfections \cite{Xu:2014:SDF}.
%
In \reffig{frankenstein-physics}, we compare to extracting the outer hull of
the multi-component and self-intersecting \emph{Frankenstein} using
\cite{Jacobson:WN:2013}.
%
While technically a ``perfectly tight fit'', this cage fails to coarsen the
domain and meshing near intersections creates sliver triangles problematic for
numerics.
%
We also compare to signed-distance field contouring \cite{Xu:2014:SDF}, which
creates a loose fit joining the legs near the knees.
%
With the same vertex count, our cage is a tight fit making it especially
suitable for elastic simulation with collision handling.
%
The deformation of the volume inside the coarse cage is then propagated via
linear interpolation to the embedded mesh (see also accompanying video).

%
\begin{wrapfigure}{r}{1.02in}
%
\centering
%
\includegraphics[trim=4mm 0 0 3mm,width=\linewidth]{figs/octopi-collisions}
%
\end{wrapfigure}
%
Another application area for nested cages is collision detection for rigid
objects: if an object does not collide against a surface enclosing a second
object, 
this certifies that the two objects do not collide either.
%
Bounding sphere, cubes, and
higher-degree polytopes are often used to quick-reject candidate collisions for
this reason, but these convex cages lose effectiveness when objects are concave
or have holes and cavities. On the other hand, a coarse nested cage is ideal
for this purpose, since the nesting property guarantees correctness and the
tight fit allows the coarse cage to efficiently reject nearly all false
positives during collision detection. As a proof of concept, we simulated
dropping eighteen instances of the octopus mesh (\reffig{zoo},
top-middle) one by one into a narrow tank, where they collide with each other
and the tank walls (\newhl{with} coefficient of restitution $0.99$) using
continuous-time collision detection (see inset). We compared the performance of
two different broad phase strategies for collision detection: in the first,
bounding cubes around swept spacetime volumes are used to prune distant pairs
of objects from consideration, then kDOP bounding volume hierarchies are built
to find candidate colliding vertex-face pairs, which are then passed to the
continuous-time narrow phase. The second strategy is the same, except
that we first check if the octopus's coarse cage is colliding before building
the BVH on the octopus itself. We found that the latter strategy is 
$\sim 8\times$ faster, for intuitive reasons: the octopus's many
protruding arms causes it to easily nestle near other copies of itself, so that
their convex hulls overlap but our tight, coarse cage does not.
%
Nesting is applicable for rigid or nearly rigid objects, but it is not obvious
to track cages along with deformable bodies, unless, for example, deformations
could be precomputed.

A single very coarse cage is needed for interactive deformation with
generalized barycentric coordinates.
%
In \reffig{dane-vs-ben-chen}, we compare to the automatic cage creation
heuristic in \cite{Ben-Chen:2009:SDT} and see much tighter fits for matching
vertex counts.

The ability to customize our optimization energy enables not just better cages,
but also better generalized barycentric coordinates. In particular, harmonic
coordinates are defined for arbitrary polyhedra, yet most works implement them
inside triangle-mesh cages only \cite{HarmonicCoodinates07}.
%
To utilize popular quad-dominant meshes as cages, all faces must be outside the
input model and \emph{planar}.
%
Since such cages are difficult to model manually, many implementations simply
triangulate high-order facets \cite{HarmonicCoodinates07}.
%
We complement the recent sketch-based quad-meshing tool \cite{Takayama:2013},
post-processing its output to enclose the input model and minimize a
planarization energy \cite{poranne2013interactive}.
%
Quad-dominant cages are easier to control as their visualization is less
cluttered (see \reffig{hand-hc}).
%
More importantly, harmonic coordinates constructed (via their recursive
definition) on the planar-quad polyhedron are also higher quality: coordinates
are smooth functions inside each quad.
%
In contrast, coordinates of a triangulated cage would depend heavily on the
choice of diagonals splitting each quad.

\begin{figure}
  \includegraphics[width=\linewidth]{figs/hand-hc}
  \caption{Left to right: a quad mesh is quickly sketched atop the \emph{Hand}
  and our pipeline moves it outside the input while planarizing quads. A
  cage-based deformation is applied via harmonic coordinates computed over the
  volume inside the planar-quad polyhedron (a few coordinates visualized in
  pseudocolor). These weights are smooth inside each quad. In contrast,
  triangulating the quads would lead to non-smooth, meshing-dependent values in
  each quad (two alternative triangulations).}
  \label{fig:hand-hc}
\end{figure}

\paragraph{Generalizations beyond watertight meshes.}
%
Though not strictly meeting our input criteria, we apply our method to
polygon soups and point clouds.
%
In \reffig{swat-cage}, we again adapt the optimization energy, this time to
maintain the reflectional symmetry of the coarse input cage.
%
The input polygon soup \emph{S.W.A.T.\ man} is riddled with meshing artifacts,
but we still flow it inside.
%
Though details on the expanding input mesh are asymmetric (see hip pockets or
hands), the energy minimization keeps the cage symmetric.

In \reffig{point-cloud}, we use Poisson surface reconstruction and decimation
to compute an initial cage for a point cloud and then nest it around the points
using our method.

\begin{figure}
  \includegraphics[width=\linewidth]{figs/point-cloud}
  \caption{Our pipeline generalizes to non-polyhedral inputs such as point
  clouds.}
  \label{fig:point-cloud}
\end{figure}

\begin{figure}
  \includegraphics[width=\linewidth]{figs/swat-cage}
  \caption{\emph{S.W.A.T. man} is a polygon soup with 2806 
  intersecting triangle pairs, 24 non-manifold edges, 51 boundary loops and 51
  components. Our shrinking flow is robust to such artifacts. Once
  inside the coarse, overlapping input cage, we re-inflate it and produce a
  fully exterior cage used to deform the embedded model.}
  \label{fig:swat-cage}
\end{figure}

\section{Limitations and future work}
\label{sec:conclusion}
%
We plan to optimize the performance of both the signed distance field flow and
the re-inflation steps.
%
Collision detection and handling dominates running time and our prototype
naively recomputes acceleration data-structures rather than updating them while
the meshes change dynamically.

Our insight to break the multi-layer nesting problem into pairwise subproblems
ensures tractability, but in some cases leads to converging at an
``artificial local minimum.'' If a coarse cage collides with itself during
inflation then it may create a pinch that blocks inflation of subsequent coarse
layers (see \reffig{homer}).
%
One solution is to iterate through the fine layers to make sufficient room in
these problem areas, but defining this relaxation direction is not obvious.

\begin{figure}
  \includegraphics[width=\linewidth]{figs/homer-fail}
  \caption{If a expanding coarse mesh collides with itself (green), it creates
  a \emph{pinch} preventing processing of further coarser layers.}
  \label{fig:homer}
\end{figure}

In cases where the input coarse cage begins too far away from the fine mesh,
the signed-distance flow will fail: for example, by flowing vertices of a
triangle into opposite parts of the coarse mesh. 
%
Adding a small amount of smoothing on the flowing fine mesh or expanding the
coarse mesh alleviates some of these problems, but a general solution is
illusive.
%
The correct assignment seems related to correctly matching medial axes of both
meshes. Perhaps this is an avenue of future improvement.

Though we are delighted with the performance of our multiresolution solver, 
we would like to further improve it. We believe there are even more gains to be
made by parameter tuning and experimenting with different coarsening
gradations. We would also like to consider using our meshes to build
multiresolution proconditioners for conjugate gradient solvers.
%
We expect that higher order PDEs with more involved boundary conditions will
receive an even greater benefit from our nested cages.

It would be interesting to analyze formally the convergence of our nested cages
along the lines of \cite{chan1996convergence} who consider the then-available
non-nested hierarchies.

Though our cages excel with in a \emph{volumetric} multiresolution solver, they
are not immediately applicable to \emph{surface-based} multiresolution problems
(cf.\ \cite{Aksoylu2005msu,Chuang:2009:ELO]}. Wether nesting is at all useful
for surface problems remains an open question.

In conclusion, nested cages prove to be a powerful tool in a variety of
applications. 
%
Our signed-distance flow consistently finds initial feasible states for our
constraint-based optimization.
%
By leveraging state-of-the-art collision handling tools from physically based
simulation, we are able to generate cages that in turn enable faster physical
simulations, more-efficient linear system solvers and better real-time
deformation user interfaces.
%
We hope that our algorithm's success encourages more multiresolution volumetric
methods using unstructured meshes in geometry processing, computer graphics,
and beyond.


\begin{figure*}
  \includegraphics[width=\linewidth]{figs/zoo}
  \caption{Each triplet shows: input model, slice through all nested layers,
  and outermost, coarsest layer. We fit 50 layers tightly around \emph{Max
  Planck}'s head (top left). The topological holes of the high-genus
  \emph{Fertility} are maintained across all layers (top right). The deep
  concavity of the \emph{Mug} does not get smoothed away in coarser levels
  (bottom right).}
  \label{fig:zoo}
\end{figure*}

\bibliographystyle{acmsiggraph}
\bibliography{references}

\section*{Appendix A: Flow counterexample}
%
\label{app:counterexample}
%
There exist polyhedra that cannot be continuously flowed inside themselves~\cite{Mathoverflow:206750}.
%
Consider the surface constructed in \reffig{vouga-fan}, right. Though inward
pointing normals can be defined at all vertices, any non-negative (even
infinitesimal) inward movement of the vertices will introduce ``edge-edge''
collisions with the original surface.
%
To see this, consider the set of points inside the polyhedron that can be "seen" by the central vertex $c$ of the star. This set changes discontinuously for any motion of $c$: no matter which direction is chosen to flow $c$, there exists some neighbor $n$ such that $c$ loses sight of not only $n$, but also an entire neighborhood around $n$. Therefore there exists no inward flow of all of the vertices where $c$ maintains sight with all of its neighbors.
\begin{figure}[hb]
\includegraphics[width=\linewidth]{figs/vouga-fan}
\caption{Take every other vertex around the star surface on the left and
rotate it about the origin to create the ``origami pinwheel'' on the right.
Vertices of this new surface cannot be continuously flowed inward without
creating intersections with the original surface.}
\label{fig:vouga-fan}
\end{figure}


\end{document}
