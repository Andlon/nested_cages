\section{Introduction}
\label{sec:introduction}

\begin{figure}[b]
  \includegraphics[width=\linewidth]{figs/horse-25-layers}
  \caption{Our tight nesting property is robust even when the number of cages
  is large: 25 around this \emph{Horse}.}
  \label{fig:horse-25-layers}
\end{figure}

As the complexity and size of computational objects continue to grow,
acceleration algorithms become increasingly important. One powerful technique
is to decompose a high-resolution mesh into a hierarchy of increasingly coarse
approximations or \emph{cages}. For example, multiresolution FEM techniques
efficiently solve Eulerian PDEs on very fine meshes by moving up and down the hierarchy; low frequency residual error disappears quickly on the coarsest levels
while fine levels smooth away high frequency error. Coarse enclosing cages also find use in physical simulation, where deformations
of the cage are interpolated onto embedded high-resolution geometries; in
interactive animation, where artists can specify large-scale deformations by
adjusting the low-dimensional cage; and in collision detection, where conservative
culling reduces computation time. For all these applications, the key to good performance is the ability to
generate a good-quality multiresolution hierarchy.

\paragraph{Desiderata} Although emphasis varies according to application, an
ideal multiresolution hierarchy possesses several properties useful across a
broad spectrum of applications (see \reftab{feature-chart}):

\begin{table}
\centering
\ra{1.2}
\setlength{\tabcolsep}{5.5pt}
\rowcolors{2}{white}{lightbluishgrey}
\begin{tabularx}{\linewidth}{X c c c}
\rowcolor{white}
\toprule
Method               & \textsc{Homeomorphic} & \textsc{Nesting} & \textsc{Tight}\\
\midrule                                  
\textbf{Our method}  & \YES               & \YES             & \YES      \\
\emph{Decimation}                         
                     & \YES               & \NO              & \YES      \\
\emph{Voxelization}                       
                     & \NO                & \YES             & \NO       \\
\emph{Level-set}                          
                     & \NO                & \YES             & \NO       \\
\bottomrule
\end{tabularx}
\caption{Hierarchy construction feature chart: Previous techniques fall short
on one property or another.}
\label{tab:feature-chart}
\end{table}

\noindent \textsc{Homeomorphic}: All levels of the hierarchy respect the
\emph{topology} of the original mesh, and do not introduce any large global
errors in the object geometry. It is important to differentiate between two
types of distances when building the hierarchy: \emph{embedded distance} in
ambient space, and \emph{intrinsic distance} within the shape.
%
It is possible for two points that are nearby in embedded distance to be very
far apart in intrinsic distance. Consider a human stretching her hands to
nearly touch her toes. The fingertips and toes may be within millimeters across
ambient space but meters apart intrinsically.
%
It is critical for many applications that coarse cages do not fuse together
intrisically-distant parts. It would be unacceptable when animating a 
human's fingers if the toes moved as well.
%
Similarly, when calculating the vibration modes of a broken ring, fusing the
ring and changing its topology would lead to largely incorrect low energy
modes.

\noindent \textsc{Nesting}:
%
Each mesh of the hierarchy completely \emph{encloses} the next-finest mesh, as
well as all finer meshes.
%
Most algorithms for transferring the pose of a deformation cage to a detailed
object only guarantee small distortion of the object and lack of
element-inversion artifacts if the object is entirely contained within the cage
\cite{HarmonicCoodinates07,Ben-Chen:2009:VHM}.
%
For accelerating collision detection, it is essential that the cage
conservatively enclose the object.
%
Nesting is not a necessary condition for multiresolution convergence
\cite{Chan96overlappingschwarz}, but the ability to use simple linear interpolation
for prolongation is known to produce more robust, more
efficient, and easier to implement solvers
\cite{chan2000robust,dickopf2010multilevel}.
%
This is particularly important for enforcing Neumann boundary conditions,
common to simulation and geometry processing \cite{chan1999boundary}.

\noindent \textsc{Tight}:
%
Each mesh in the hierarchy fits \emph{tightly} around the next-finest level of
the hierarchy.
%
Tightness can be quantified by the volume or Hausdorff distance between cages.
%
Across all applications, a tight fit minimizes the geometric error at the
object boundary incurred by simplifying the input mesh.

Existing popular approaches to construct coarse cages satisfy some, but not
all, of these three properties.

\emph{Decimation} of the original surface mesh may yield coarse
approximations fitting very tightly to the original and preserving detail.
%
Constrained decimations also preserve the topology of the original mesh, but in
general a decimation will intersect the original mesh nearly everywhere, rather
than encage it.

\emph{Voxelization} of the object to a regular grid is guaranteed to produce
an enclosing cage, but the fit will not be tight and intrinsically distant
features can be fused together if the grid is too coarse. Adaptive grids
(octrees, etc.) improve the fit and topological fidelity, but unless the object
contains mostly axis-aligned features the resulting cage will not efficiently
fit the object.

\emph{Level-set} methods can ensure nesting, but can fuse spatially proximate
features, changing the object topology. Strict nesting, and in turn tightness,
are also in danger when using standard contouring as output surface meshes will
cut corners in high curvature areas, leading to intersections with other
cages.

\alec{need to reference teaser in intro}

\paragraph{Our Contribution}
%
We propose an algorithm for computing a
multiresolution hierarchy that satisfies all three of the desired properties.
%
Given an input mesh, we decimate the model to the desired levels while
preserving its original topology.
%
We then recursively deform each fine cage to contain the next finest output
cage.
%
Working with a consecutive pair of cages,
we shrink the fine cage along a signed-distance-field flow until inside the
coarse cage.
%
We then rewind the flow, returning the fine cage to its original positions.
During this \emph{re-inflation}, we run collision detection and handling
\emph{à la} physical simulation to update the positions of the coarse
containing cage. Contact forces guarantee that the coarse cage always encloses
the reversing fine mesh at each reverse-flow time step.
%
When complete, the output coarse cage takes the role of the fine cage for the
next coarsest layer.

During re-inflation we optimize the positions of the coarse cage by minimizing
a user-specified, problem-specific energy: for example, total volume for tight fitting
cages for efficient multiresolution solvers, as-rigid-as-possible (ARAP) energy for
minimal distortion for simulation or collision handling, or symmetry and quad
planarity for automatically computing interactive deformation cages.

% Multiresolution hierarchies
  % tight fitting, enclosing, same topology

% simulation meshes
  % respect semantically distant parts

% cages for deformation
  % symmetric cages for overall symmetric shapes with asymmetric details
  % planarity, first to show harmonic coordinates using quad cages

% Coarse cages for conservative collision detection
