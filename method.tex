\section{Method}
\label{sec:method}

The input of our method is a closed orientable triangle mesh in $\mathbb{R}^3$:
a set of $n$ vertices $\mathbf{V}_0= \{ \mathbf{v}_1, \ldots, \mathbf{v}_n \}
\subset \mathbb{R}^3$ and a set of $m$ facets $\mathbf{F}_0 = \{ f_1, \ldots,
f_m \} \subset \{ 1, \ldots, n \}^3$.  Following the desiderata discussed in
section \ref{sec:introduction}, we want the output of our method to be a set of
meshes $(\mathbf{V}_1,\mathbf{F}_1)$,  $(\mathbf{V}_2,\mathbf{F}_2)$, $\ldots$,
$(\mathbf{V}_N,\mathbf{F}_N)$ such that 

\begin{enumerate}[(i)]
\item $(\mathbf{V}_{i},\mathbf{F}_{i})$ completely encloses
$(\mathbf{V}_j,\mathbf{F}_j)$, for all $i = 1, \ldots, N$ and $j = 0, \ldots,
i-1$.
\item $(\mathbf{V}_{i},\mathbf{F}_{i})$ is topologically equivalent and
geometrically similar to $(\mathbf{V}_{0},\mathbf{F}_{0})$, for all $i = 1,
\ldots, N$.
\item $(\mathbf{V}_{i},\mathbf{F}_{i})$ tightly fits
$(\mathbf{V}_{i-1},\mathbf{F}_{i-1})$, for all $i = 1, \ldots, N$.
\end{enumerate}

Let $(\mathbf{V}_i,\mathbf{F}_i)$ be a level already computed. To compute the
next level $(\mathbf{V}_{i+1},\mathbf{F}_{i+1})$ we first perform
\emph{topology-aware decimation} on $(\mathbf{V}_i,\mathbf{F}_i)$ (figure XXX),
then move $(\mathbf{V}_i,\mathbf{F}_i)$ towards inside this decimated mesh
using a \emph{geometric flow} specific for the task of shrinking one mesh
towards inside another (figure XXX). 

Once the finer mesh is completely inside the coarser one, we expand it back to
its original positions (given by $\mathbf{V_i}$) while performing
collision-aware energy \emph{minimization over the coarse mesh} in a way it
ends up enclosing $(\mathbf{V}_i,\mathbf{F}_i)$ (figure XXX). The expansion of
the fine mesh defines constraints that impose the coarser mesh to be exterior
to it, while the energy measures similarity and tight fit between both meshes.

After all the nested layers are generated in this pairwise manner, we perform
collision-aware \emph{global  minimization} over all meshes, in order to
improve tight fit and similarity of the set of meshes as a whole. 
\\ 
\\
\textbf{Topology-aware decimation:} \leo{Problem - Qslim is not topology-aware,
according to design goals in section 3.1 of Garland's thesis.} Mention number
of faces of the output meshes... 
\\ 
\\ 
\textbf{Geometric flow:} After computing
an initial decimation $(\mathbf{U}_{i+1},\mathbf{G}_{i+1})$ for an already
computed level $(\mathbf{V}_{i},\mathbf{F}_{i})$, we now want to shrink
$(\mathbf{V}_{i},\mathbf{F}_{i})$ completely inside
$(\mathbf{U}_{i+1},\mathbf{G}_{i+1})$ (figure XXX). When this is achieved, fine
and coarse layers no longer intersect and we will be able to inflate the fine
level to its original positions $\mathbf{V}_{i}$ while pushing the coarse level
away via a physical simulation (figure XXX). The coarse mesh resulting from
this process will be a cage for $(\mathbf{V}_{i},\mathbf{F}_{i})$. 

Instead of pursuing a single mesh that is totally inside
$(\mathbf{U}_{i+1},\mathbf{G}_{i+1})$, we opt for defining a path of meshes
$(\Phi_{t_k},\mathbf{F}_{i})$ that move inwards the coarse mesh since reversing
the path stepwise will give the physical simulation a more tractable set of
constraints per step. There are many geometric flows in the literature with
shrinking properties (\leo{cite MCF, cMCF, Willmore}), but they present some
drawbacks for our task. First of all, there is no guarantee that they end up
completely inside the coarse mesh. Second, these flows smooth the surface's
geometry, leading to a loss of features along their way. This can impose
excessive constraints for the physical simulation since more collisions will
happen when trying to inflate the mesh to its original positions. The simpler
solution of finding a point completely inside
$(\mathbf{U}_{i+1},\mathbf{G}_{i+1})$ and shrinking
$(\mathbf{V}_i,\mathbf{F}_{i})$ linearly towards it would present this last
drawback. \leo{Would it be nice to put a (maybe 2D) figure illustrating all
these claims?} 

The flow that most closely reaches our goals of shrinking one mesh inside the
other following geometric features is \leo{Tagliasachhi's}. But this flow has
two additional drawbacks: first, it heavily relies on remeshing. We need the
meshes $(\Phi_{t_k},\mathbf{F}_{i})$ to have the same connectivity to be able
to use them on the physical simulation. Moreover, \leo{Tagliasachhi's} moves
the fine mesh towards inside its own medial axes and this is not guaranteed to
end up totally inside the coarse mesh \leo{Figure illustrating this last point
would also be cool.} 

Ideally we would like the fine mesh to be totally inside the coarse at some
point. Since none of the options above fulfil this requirement, we define a
proper flow for doing this. In the continuous setting, we seek a flow that
contracts a given surface $M$ inside a closed orientable surface $S$ with a
fixed embedding $\Psi: S \to \mathbb{R}^3$. Let $D: \mathbb{R}^3 \to
\mathbb{R}$ be the signed distance field induced by $\Psi(S)$
\begin{align}
D(\mathbf{p}) = \left\{ \begin{array}{rl} 
\mathrm{dist}(\mathbf{p},\Psi(S)), & \mbox{if } \mathbf{p} \mbox{ is outside } \Psi(S) 
\\
-\mathrm{dist}(\mathbf{p},\Psi(S)), & \mbox{if } \mathbf{p} \mbox{ is inside } \Psi(S)
\end{array}
\right. ,
\end{align}
where $\mathrm{dist}$ is the point-set Euclidean distance. We can contract $M$
inwards $\Psi(S)$ by finding an embedding $\Phi(M)$ that minimizes
\begin{align}
E(\Phi) = \int_M D(\Phi(\mathbf{p}))\ d \mu_0,
\label{eq:distance_energy}
\end{align} 
where $d \mu_0$ is a \emph{fixed} metric on $M$. We fix the metric in
(\ref{eq:distance_energy}) because otherwise the minimization would focus on
minimising both the distance field and the surface meteoric, while our goal is
to only minimize the distance filed. \leo{Figure of a 2d example to illustrate
the importance of fixing the metric.}

There are no constraints for the embedding $\Phi(M)$, so minimizing $E$ in
(\ref{eq:distance_energy}) is an unconstrained minimization problem that can be
minimized using the gradient method with line search respecting Armijo's
criterium (\leo{reference needed?}). Let $(\mathbf{V}, \mathbf{F})$ be a
triangle mesh that represents $\Phi(M)$ and consider the following quadrature
discretization of  (\ref{eq:distance_energy}):

\textbf{Minimization over the coarse mesh:}
\\
\\
\textbf{Global minimization:}

\section{Implementation}
\label{sec:implementation}


